%%%%%%%%%%%%%%%%%%%%%%%%%%%%%%%%%%%%%%%%%%%%%%%%%%%%%%%%%%%%%
% ECE 445 SENIOR DESIGN TEMPLATE
%%%%%%%%%%%%%%%%%%%%%%%%%%%%%%%%%%%%%%%%%%%%%%%%%%%%%%%%%%%%%
\documentclass[openbib,letterpaper,10pt]{article}

%%%%%%%%%%%%%%%%%%%%%%%%%%%%%%%%%%%%%%%%%%%%%%%%%%%%%%%%%%%%%
% The preamble starts here.
% You can add other packages that you want to use by using
% \usepackage command in the preamble.
% However, DO NOT change the settings that are already placed
% below unless you really know what you are doing.
%%%%%%%%%%%%%%%%%%%%%%%%%%%%%%%%%%%%%%%%%%%%%%%%%%%%%%%%%%%%%

% some commonly used packages
\usepackage{graphicx}
\usepackage{color,soul}
\usepackage{amsmath}
\usepackage{amsthm}
\usepackage{amsfonts}
\usepackage{setspace}
\usepackage{longtable}
\usepackage{url}
\usepackage{float}
\usepackage{caption}
\usepackage[colorlinks=true,linkcolor=black,citecolor=black]{hyperref}
\usepackage[top=1in, bottom=1in, left=1in, right=1in]{geometry}% set the page margins to 1 inch

% use the fancyhdr package to maintain the format of the page numbers,
% which is useful when the text color is changed
\usepackage{fancyhdr}
\fancyhf{}
\renewcommand{\headrulewidth}{1pt}
\renewcommand{\footrulewidth}{0pt}
\fancyfoot[C]{\textcolor{black}{\thepage}}
\fancyhead[L]{\includegraphics[width=2cm]{University-of-Illinois-logo.jpg}}
\fancyhead[R]{\small{Infantry I.F.F. Mock Design Review - Meyers \& Prince}}

% paralist provides extended list environments
\usepackage{paralist}
\setlength{\plitemsep}{0pt}

% define the color for section and subsection titles
\usepackage{xcolor}
\definecolor{titlecolor}{RGB}{31,73,125}
\definecolor{subtitlecolor}{RGB}{79,129,189}

% tikz/pgf environment for making graphs
\usepackage{tikz}
\usepackage{tikz-timing}[2009/12/09]
\usetikzlibrary{shapes,arrows}
\usetikztiminglibrary[new={char=Q,reset char=R}]{counters}

% change the style of the abstract environment
\usepackage{abstract}
\setlength{\absparsep}{6pt}
\setlength{\absleftindent}{0pt}
\setlength{\absrightindent}{0pt}
\setlength{\abstitleskip}{-18pt}
\renewcommand{\absnamepos}{flushleft}
\renewcommand{\abstractnamefont}{\normalfont\Large\singlespacing\bfseries}
\renewcommand{\abstractname}{\textcolor{titlecolor}{Abstract}}

% change the style of the section and subsection titles
\usepackage{titlesec}
\titleformat{\section}{\color{titlecolor}\Large\bf}{\color{titlecolor}\thesection}{0.8em}{}
\titleformat{\subsection}{\color{subtitlecolor}\large\bf}{\color{subtitlecolor}\thesubsection}{1em}{}
\titleformat{\subsubsection}{\color{subtitlecolor}\normalsize\bf}{\color{subtitlecolor}\thesubsubsection}{1.2em}{}
\titlespacing{\section}{0pt}{0em}{0em}
\titlespacing{\subsection}{0pt}{0em}{0em}
\titlespacing{\subsubsection}{0pt}{0em}{0em}

% change the style of the table of contents
\usepackage{titletoc}
\titlecontents{section}[1.5em]{}{\contentslabel{1.5em}}{\hspace*{-1.5em}}{\titlerule*[0.5pc]{.}\contentspage}
\titlecontents{subsection}[3em]{}{\contentslabel{2.1em}}{\hspace*{-2.1em}}{\titlerule*[0.5pc]{.}\contentspage}
\titlecontents{subsubsection}[5.1em]{}{\contentslabel{2.7em}}{\hspace*{-2.7em}}{\titlerule*[0.5pc]{.}\contentspage}

% command for centering texts in a fixed width table cell
\newcommand{\centpcol}{\leftskip\fill \rightskip\fill}

% command for setting the style of the appendix titles
\newcommand{\setappenstyle}{
	\titleformat{\section}{\color{titlecolor}\Large\bf}{\color{titlecolor}Appendix \Alph{section}}{0.8em}{}
	\titlecontents{section}[0em]{}{Appendix \thecontentslabel \hspace{1em}}{}{\titlerule*[0.5pc]{.}\contentspage}
}

% define the style of the title of the paper
\newcommand{\thetitle}[1]{\title{\begin{huge}{\bf #1}\end{huge} \color{subtitlecolor}\rule[25pt]{\textwidth}{1pt}}}

% define the style of the author
\newcommand{\theauthor}[3]{
	\author{\vspace{.4in}\\
	\textcolor{black}{By}\\
	#1
	\vspace{1in}\\
	\textcolor{black}{ECE 445 Mock Design Review -} #2\\
	\textcolor{black}{TA:} #3
	\vspace{1in}}
}

% define the style of figure's caption
\newcommand{\figcap}[1]{
	\captionsetup{format=plain,font={small,color=subtitlecolor,singlespacing},margin={0pt,0pt}}
	\caption{\textcolor{subtitlecolor}{#1}}
	\vspace{-5pt}
}

% define the style of table's caption
\newcommand{\tablecap}[1]{
	\captionsetup{format=plain,font={bf,normalsize,singlespacing,color=black},margin={0pt,0pt}}
	\caption{\textcolor{black}{#1}}
	\vspace{-5pt}
}

% define the style of the abstract's page number
\newcommand{\abstractsetting}{
	\pagenumbering{roman}
	\thispagestyle{fancy}
}

\newcommand{\buildtoc}{
	\clearpage
	\singlespacing
	\tableofcontents
	\onehalfspacing
}

% set indentations and the space between paragraghs
\setlength{\parindent}{0pt}
\setlength{\parskip}{8pt}

%%%%%%%%%%%%%%%%%%%%%%%%%%%%%%%%%%%%%%%%%%%%%%%%%%%%%%%%%%%%%
% PREAMBLE ENDS HERE, DOCUMENT STARTS BELOW
%%%%%%%%%%%%%%%%%%%%%%%%%%%%%%%%%%%%%%%%%%%%%%%%%%%%%%%%%%%%%

\begin{document}

% don't change these
\pagestyle{empty}
\doublespacing

% put the title of your project here. DO NOT include the brackets.
\thetitle{{I.F.F. (Identification Friend or Foe) System}}

% put your names here. seperate by \\. DO NOT include the brackets.
\theauthor{
	{Eric Meyers (emeyer7)}\\
	{Noah Prince (nprince2)}\\
}
{ % put the semester info here. DO NOT include the brackets.
	{Spring 2016}
}
{ % put your TA's name here. DO NOT include the brackets.
	{Braedon Salz}
}

% put the date and project number here. DO NOT include the brackets.
\date{
{February 18th, 2016}\\
Project No. 11
\clearpage
}

% don't change these
\maketitle
\pagestyle{fancy}
\begin{spacing}{1.15}


% build the table of contents. 
\color{black}
\buildtoc
\pagenumbering{gobble}
\section*{Acronyms \& Pre-Requisite Information}
\begin{itemize}
	\item MCU - Microcontroller Unit
	\item R.F. - Radio Frequency
	\item T.I. - Texas Instruments
	\item 
\end{itemize}
\clearpage
\setcounter{page}{1}
\pagenumbering{arabic}

%SECTION 1 - Introduction - Eric
\section{Introduction}
This document is a "Mock Design Review" in preparation for the Design Review occuring during the week of February 29th, 2016. This will better prepare the team for documentation of the design and construction of the Infantry I.F.F. System.

%SECTION 2 - Design - Eric
\section{Block Diagram}
\begin{figure} [H]
	\centering
	\includegraphics[scale=0.50]{Block_Diagram.png}
	\caption{Block Diagram of Laser Transmitter\label{fig:circuit-schematic}}
\end{figure}

%SECTION 5 - Block Description - Eric
\section{Block Description}
The subsystem of the Laser Transmitter will be broken down into 5 primary modules:
\begin{enumerate}
	\item Power Module
	\item Voltage Step Down Regulator
	\item Microcontroller 
	\item Real Time Clock
	\item Laser Diode
\end{enumerate}

\subsection*{Power Module}
The Power Module will consist of a standard 9V battery. The type of 9-volt battery is at the discretion of the operator due to the availability on the market (i.e. either rechargable or disposable). However, a battery with least 250 mAh of use time must be selected to supply the circuit with 9V over a period of 8 hours. The team will chose to use a 9V 300 mAh NiMH Rechargable Battery for testing purposes.

The team decided to use a 9 V battery instead of four double-A batteries due to the need of maintaining a constant 3.3V over time as well as simplicity (having one battery with a regulator is much simpler than having 4 double A batteries).

\subsection*{Voltage Step Down Regulator}
The LD1117V33 voltage step-down regulator will take the 9V supply input and step it down to 3.3V to supply the MSP430 MCU. The voltage regulator will supply a maximum of 900 mA of current which will be significantly less than this circuit draws.

\subsection*{Microcontroller}
This design choice was by far the most difficult. The team chose to work with an T.I. MSP430F2274 Microcontroller Unit due to its simplicity, its availability in the ECE445 Senior Design Labs (inventory) and the number of GPIO Pins on board. The 

\subsection*{Real Time Clock}
The Real Time Clock is not entirely neccessary for the operation of the Laser Transmitter Subsystem, however it will be neccessary for the operation of the R.F. Receiver and thus must be included in the MCU circuit. It will operate using a 32.768 kHz Crystal Oscillator (as reccomended by T.I.)

\subsection*{Laser Diode}
The 5mW laser diode will operate on 3.3V at 25mA so a 1.3l$\omega$ resistor is neccessary to drop the current being supplied to the diode down to this threshold. This laser diode will prode a beam width of \_\_\_

Due to safety and ethical considerations, the requirements have changed for the divergence of the beam. The proposal stated a requirement of a 5-6 ft diameter beam at 50, 150, and 300 m (with optical adjustments allowed). 

With some tolerance, the PIN photodiode can register an irradiance of 9 $\frac{W}{m^2}$. The laser required to achieve this irradiance is a function of the laser power and the radius of the beam. With relatively short distances of 0-300 m, atmospheric deflection of light is negligible. 

Using trigonometry, the power as a function of radius is given by $P = \pi r^2  E_{req} $, where $r$ is the radius (in meters) and $E_{req}$ is the required irradiance of 9 $\frac{W}{m^2}$. 

For a diameter equal to the one stated in the proposal ($\approx 1.6764 m$), a $\pi (0.8382)^2(9) \approx 20 mW$ laser is needed. A $20 mW$ laser is a Class 3B laser, and is considered dangerous. For the scope of this senior design project, a $5mW$ laser will be used instead. 

The radius achieved using a $5 mW$ laser is given by $\sqrt{\frac{P}{E_{req} \pi}} = 0.420522 m$. This is approximately a $2.75 ft$ diameter beam, which is still the size of a person's chest. 

\clearpage

%SECTION 3 - Circuit Schematic - Eric
\section{Circuit Schematic}
\begin{figure} [H]
	\centering
	\includegraphics[scale=0.38]{Circuit_Schematic.png}
	\caption{Circuit Schematic of Laser Transmitter\label{fig:circuit-schematic}}
\end{figure}

%SECTION 4 - Plot/Experiment - Noah
\section{Plot}
NOAH SECTION

%SECTION 5 - Requirements and Verification - Both 
\section{Requirements and Verification}
FILL OUT TOGETHER

%SECTION 6 - Safety - Noah
\section{Safety \& Ethical Considerations}
The proposal requirements would have required a $20mW$ IR laser. \hl{Eric, can you put more information about registering the laser and shit at 20mW?}. 

The team will instead use a $5mW$ visible red laser. $5mW$ visible lasers have a low chance of injuring the eye, as the blinking reflex will save a victim from permanent damage; as opposed to IR lasers which can go unnoticed for several seconds. 

The following is a calculation for the nominal ocular hazard distance (NOHD) of our laser, as defined by the ANSI Standard \cite{ANSI}.

The maximum permissible exposure (MPE), as defined by the ANSI Standard \cite{ANSI} is the highest power or energy density of a light source that is considered safe, i.e. that has a negligible probability for creating damage. This MPE for a pulsing laser is calculated as the minimum of the following three rules:

\begin{enumerate}
	\item Any single pulse in the train must not exceed the MPE for the pulse exposure time.
	\item The exposure from any group of pulses delivered in time T must not exceed the MPE for
	time T, where T is 0.25 seconds (from the blinking reflex), for a visible laser. 
	\item For thermal injury, the exposure for any single pulse within a group of pulses must not
	exceed the single-pulse MPE multiplied by a multiple-pulse correction factor
\end{enumerate}

The laser will pulse at a rate of 40 kHz. Assuming at most a 50\% duty cycle, each pulse will be of max length $1.25*10^{-5} s$. The divergence of the beam is smallest for the longest range; a lower divergence is more restrictive in terms of safety, so this calculation uses $300m$. The divergence of the beam for 300m is 2.79 mrad and the beam waist is approximately $4 mm$. \\

Following the ANSI Standard \cite{ANSI}, the Rule 1 calculation is 

{\large $5*10^{-3}*(\frac{2.79}{1.5})  = 0.0093 \frac{J}{m^2}$ }\\

The Rule 2 calculation is

{\large $\frac{18(.25^{0.75})(\frac{2.79}{1.5})}{.25*40000} = 0.0011837 \frac{J}{m^2}$ }\\

The Rule 3 calculation is

{\large $(.25*40000)^{0.25} * 5*10^{-3}*(\frac{2.79}{1.5}) = 0.093 \frac{J}{m^2}$ }\\

The most restrictive of all the rules is Rule 2, which gives us an MPE of $0.0011837 \frac{J}{m^2}$.

At $5mW$ with a pulse width of $1.25*10^{-5}$, the power of the laser is $6.25*10^{-8} J$. 

The NOHD is defined as

{\LARGE $ \frac{\sqrt{\frac{4 * P}{\pi * MPE}} - 2w}{\theta}$}

Where P is the power of the beam ($6.25*10^{-8} J$) and $w$ is the waist of the beam ($1mm$). This gives an NOHD of 

{\Large$ \frac{\sqrt{\frac{4 * 6.25*10^{-8}}{\pi * 0.0011837}} - 2(0.004)}{0.00279} = 0.0713 m \approx 3 in $}
	
The team will avoid eye damage by not working with their eyes inside of 3 inches from the laser. If it is necessary to get this close to the laser, the team will wear eye protection or simply power off the laser. 

\clearpage
\bibliographystyle{IEEE_ECE}
% include the BibTex file here to build reference
\bibliography{Citations}\addcontentsline{toc}{section}{Reference}

\clearpage
\end{spacing}
\end{document}

