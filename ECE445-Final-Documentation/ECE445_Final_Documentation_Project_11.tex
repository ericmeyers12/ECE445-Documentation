%%%%%%%%%%%%%%%%%%%%%%%%%%%%%%%%%%%%%%%%%%%%%%%%%%%%%%%%%%%%%%
% ECE 445 SENIOR DESIGN TEMPLATE
%%%%%%%%%%%%%%%%%%%%%%%%%%%%%%%%%%%%%%%%%%%%%%%%%%%%%%%%%%%%%
\documentclass[letterpaper,10pt]{article}

%%%%%%%%%%%%%%%%%%%%%%%%%%%%%%%%%%%%%%%%%%%%%%%%%%%%%%%%%%%%%
% The preamble starts here.
% You can add other packages that you want to use by using
% \usepackage command in the preamble.
% However, DO NOT change the settings that are already placed
% below unless you really know what you are doing.
%%%%%%%%%%%%%%%%%%%%%%%%%%%%%%%%%%%%%%%%%%%%%%%%%%%%%%%%%%%%%

% some commonly used packages
\usepackage{siunitx}
\usepackage{graphicx}
\usepackage{color,soul}
\usepackage{amsmath}
\usepackage{amsthm}
\usepackage{amsfonts}
\usepackage{setspace}
\usepackage{longtable}
\usepackage{url}
\usepackage{pdfpages}
\usepackage{float}
\usepackage{rotating}
\usepackage{caption}
\usepackage{booktabs}  % professional-looking tables
\usepackage{multicol} %used for getting multicolumn without page-break
\usepackage{multirow}	% multi-row tables
\usepackage{array}		% define column format of a table
\usepackage[colorlinks=true,linkcolor=black,citecolor=black]{hyperref}
\usepackage[top=1.1in, bottom=1.1in, left=1.1in, right=1.1in]{geometry}% set the page margins to 1 inch
\usepackage{amsmath}
\usepackage{algorithm}
\usepackage[noend]{algpseudocode}

% use the fancyhdr package to maintain the format of the page numbers,
% which is useful when the text color is changed
\usepackage{fancyhdr}
\fancyhf{}
\renewcommand{\headrulewidth}{1pt}
\renewcommand{\footrulewidth}{0pt}
\fancyfoot[C]{\textcolor{black}{\thepage}}
\fancyhead[L]{\includegraphics[width=2cm]{University-of-Illinois-logo.jpg}}
\fancyhead[R]{\small{Infantry I.F.F. Final Report - Meyers \& Prince}}

% paralist provides extended list environments
\usepackage{paralist}
\setlength{\plitemsep}{0pt}

% define the color for section and subsection titles
\usepackage{xcolor}
\definecolor{titlecolor}{RGB}{31,73,125}
\definecolor{subtitlecolor}{RGB}{79,129,189}

% change the style of the section and subsection titles
\usepackage{titlesec}
\titleformat{\section}{\color{titlecolor}\Large\bf}{\color{titlecolor}\thesection}{0.8em}{}
\titleformat{\subsection}{\color{subtitlecolor}\large\bf}{\color{subtitlecolor}\thesubsection}{1em}{}
\titleformat{\subsubsection}{\color{subtitlecolor}\normalsize\bf}{\color{subtitlecolor}\thesubsubsection}{1.2em}{}
\titlespacing{\section}{0pt}{0em}{0em}
\titlespacing{\subsection}{6pt}{0em}{0em}
\titlespacing{\subsubsection}{12pt}{0em}{0em}



% change the style of the table of contents
\usepackage{titletoc}
\titlecontents{section}[1.5em]{}{\contentslabel{1.5em}}{\hspace*{-1.5em}}{\titlerule*[0.5pc]{.}\contentspage}
\titlecontents{subsection}[3em]{}{\contentslabel{2.1em}}{\hspace*{-2.1em}}{\titlerule*[0.5pc]{.}\contentspage}
\titlecontents{subsubsection}[5.1em]{}{\contentslabel{2.7em}}{\hspace*{-2.7em}}{\titlerule*[0.5pc]{.}\contentspage}

% command for centering texts in a fixed width table cell
\newcommand{\centpcol}{\leftskip\fill \rightskip\fill}

% command for setting the style of the appendix titles
\newcommand{\setappenstyle}{
	\titleformat{\section}{\color{titlecolor}\Large\bf}{\color{titlecolor}Appendix \Alph{section}}{0.8em}{}
	\titlecontents{section}[0em]{}{Appendix \thecontentslabel \hspace{1em}}{}{\titlerule*[0.5pc]{.}\contentspage}
}

\makeatletter
\newcommand{\skipitems}[1]{%
	\addtocounter{\@enumctr}{#1}%
}

% define the style of the title of the paper
\newcommand{\thetitle}[1]{\title{\begin{huge}{\bf #1}\end{huge} \color{subtitlecolor}\rule[25pt]{\textwidth}{1pt}}}

% define the style of the author
\newcommand{\theauthor}[3]{
	\author{\vspace{.4in}\\
	\textcolor{black}{By}\\
	#1
	\vspace{1in}\\
	\textcolor{black}{ECE 445 Final Report -} #2\\
	\textcolor{black}{TA:} #3
	\vspace{1in}}
}

% define the style of figure's caption
\newcommand{\figcap}[1]{
	\captionsetup{format=plain,font={small,color=subtitlecolor,singlespacing},margin={0pt,0pt}}
	\caption{\textcolor{subtitlecolor}{#1}}
	\vspace{-5pt}
}

% define the style of table's caption
\newcommand{\tablecap}[1]{
	\captionsetup{format=plain,font={bf,normalsize,singlespacing,color=black},margin={0pt,0pt}}
	\caption{\textcolor{black}{#1}}
	\vspace{-5pt}
}


\newcommand{\buildtoc}{
	\clearpage
	\singlespacing
	\tableofcontents
	\onehalfspacing
}

% set indentations and the space between paragraghs
\setlength{\parindent}{0pt}
\setlength{\parskip}{8pt}

\setcounter{secnumdepth}{4}

\titleformat{\paragraph}
{\normalfont\small\bfseries\color{subtitlecolor}}{\theparagraph}{1em}{}
\titlespacing*{\paragraph}
{18pt}{3.25ex plus 1ex minus .2ex}{1.5ex plus .2ex}

%%%%%%%%%%%%%%%%%%%%%%%%%%%%%%%%%%%%%%%%%%%%%%%%%%%%%%%%%%%%%
% PREAMBLE ENDS HERE, DOCUMENT STARTS BELOW
%%%%%%%%%%%%%%%%%%%%%%%%%%%%%%%%%%%%%%%%%%%%%%%%%%%%%%%%%%%%%

\begin{document}

% don't change these
\pagestyle{empty}
\doublespacing

% put the title of your project here. DO NOT include the brackets.
\thetitle{{I.F.F. (Identification Friend or Foe) System}}

% put your names here. seperate by \\. DO NOT include the brackets.
\theauthor{
	{Eric Meyers (emeyer7)}\\
	{Noah Prince (nprince2)}\\
}
{ % put the semester info here. DO NOT include the brackets.
	{Spring 2016}
}
{ % put your TA's name here. DO NOT include the brackets.
	{Brady Salz}
}

% put the date and project number here. DO NOT include the brackets.
\date{
{May 4th, 2016}\\
Project No. 11
\clearpage
}

% don't change these
\maketitle
\pagestyle{fancy}
\begin{spacing}{1.15}


% build the table of contents. 
\color{black}
\pagenumbering{gobble}
\section*{Abstract}
This project is a reliable method to determining the status of friendly or enemy soldiers during combat. This is an attempt to reduce the number of friendly fire or misfire accidents between soldiers on foot. A single LED indicator lights up identifying the target as either ``friendly" or ``enemy".
\buildtoc
\pagenumbering{gobble}
\clearpage
\setcounter{page}{1}
\pagenumbering{arabic}

%SECTION - Introduction
\section{Introduction}
The purpose of this project is to create a system that quickly and accurately identifies friendly targets among military personnel on foot. Similar systems exist for aircraft, however not many exist for infantry.

The idea is to develop a two-way communication system so that when a soldier aims their weapon in the direction of a friendly target, they will receive notification through an LED that the target is friendly and not an enemy. Throughout this document the infantry unit with the weapon will be referred to as the ``friendly interrogator" and the target will  be referred to as the ``friendly target". 

\subsection{Objectives}
\subsubsection{Goals and Benefits}
\begin{itemize}
	\item Reduce the number of friendly fire \& misfire accidents during combat.
	\item Notify friendly personnel of friendly target when aiming in their direction.
	\item Other applications include paintball, airsoft arcade laser tag, and various recreation sports.
\end{itemize}


\subsubsection{Functions and Features}
\begin{itemize}
	\item Laser transmitter on friendly interrogator to send unique I.D.
	\item Photodiodes on friendly target detect unique I.D. and verify its signal.
	\item R.F. transmitter on friendly target to send acknowledgement back to interrogator.
	\item R.F. receiver on friendly interrogator to verify that the target is friendly.
	\item LED to indicate friendly or enemy on interrogator unit with system response of less than 190 ms (human reaction time\textsuperscript{\cite{Reaction_Times}})
\end{itemize}

These functions and features are summarized in the system block diagram shown in Figure \ref{fig:system-block-diagram}. 

\begin{figure} [H]
	\centering
	\includegraphics[scale=0.45]{System_Block_Diagram.png}
	\caption{System Block Diagram\label{fig:system-block-diagram}}
\end{figure}

The two-way communication system on both units is further divided into two one-way communication channels. The laser transmitter on board the friendly interrogator sends a signal to the photoreceivers on the friendly target. The R.F. transmitter on board the friendly target will then send acknowledgement back to the friendly interrogator. 

An important aspect of this project is encryption and ensuring an enemy cannot pose as friendly to the interrogator. This is addressed in two ways. Both systems will contain a locally synced clock so that a friendly target can only validate their acknowledgement within a period of time. Second, the acknowledgment that is sent back to the interrogator is a function of a common passphrase only know to the two systems.


%SECTION - DESIGN
\section{Design}

%DESIGN PROCEDURE
\subsection{Design Procedure} 

\subsubsection{Friendly Interrogator}
\begin{figure} [H]
	\centering
	\includegraphics[scale=0.1]{interrogator_picture.png}
	\caption{Friendly Interrogator Unit\label{fig:interrogator-picture}}
\end{figure}

\hspace{5mm}\textbf{Laser Transmitter} \label{section:laser-transmitter-design-procedure}

For safety reasons, the maximum allowable power for the laser diode is $5mW$; which registers as a Class IIIa laser. The laser diode must also fall in the visible range, so that it will trigger a person's blinking reflex before eye damage occurs. Specifically, the team used a 5 mW 635 nm laser diode to transmit the unique I.D. (as specified by the 8-pin DIP switch) to the friendly target. 

This laser was placed inside of a casing made out of hollow aluminum tube that was created to allow the user to adjust the spot size on the laser dot. The laser can be seen in Figure \ref{fig:interrogator-picture} The laser diode, sourced by a transistor, allows for pulsing of a unique identification number at $5kHz$. The limiting factor for the $5kHz$ requirement was the processing speed of the MSP430 microcontroller; between every sampling of the photoreceiver, a significant amount of processing must occur. 

\hspace{5mm}\textbf{Laser Safety}

\hl{This stuff was just copy-pasted from the design review - probably will need to change it up a bit}

 To achieve the laser beam diameter at $300 m$ associated in the proposal, a Class 3B laser would be required. In the State of Illinois, a Class 3B laser must be registered with the Division of Nuclear Safety in the Illinois Emergency Management Agency. The 3B laser would also present a significant viewing hazard; especially in an application where the laser is intended to be pointed at people. 
 
 For the reasons stated above, the team will instead use a $5mW$ visible red laser. $5mW$ visible lasers have a low chance of injuring the eye, as the blinking reflex will save a victim from permanent damage; as opposed to IR lasers which can go unnoticed for several seconds. 
 
 The following is a calculation for the nominal ocular hazard distance (NOHD) of the laser, as defined by the ANSI Standard\textsuperscript{\cite{ANSI}}.
 
 The maximum permissible exposure (MPE), as defined by the ANSI Standard \textsuperscript{\cite{ANSI}} is the highest power or energy density of a light source that is considered safe, i.e. that has a negligible probability for creating damage. This MPE for a pulsing laser is calculated as the minimum of the following three rules:
 
 \begin{enumerate}
 	\item Any single pulse in the train must not exceed the MPE for the pulse exposure time.
 	\item The exposure from any group of pulses delivered in time T must not exceed the MPE for
 	time T, where T is 0.25 seconds (from the blinking reflex), for a visible laser. 
 	\item For thermal injury, the exposure for any single pulse within a group of pulses must not
 	exceed the single-pulse MPE multiplied by a multiple-pulse correction factor
 \end{enumerate}
 
 The laser will pulse at a rate of $40 kHz$. Assuming at most a 50\% duty cycle, each pulse will be of max length $1.25*10^{-5} s$. The divergence of the beam is smallest for the longest range; a lower divergence is more restrictive in terms of safety, so this calculation uses $300m$. 
 
 At $5mW$ with a pulse width of $1.25*10^{-5}$, the power of the laser is $6.25*10^{-8} J$. 
 
 ANSI defines several constants for use in the calculation of laser safety. The relevant constant for these calculations is the constant $C_6$. This is defined as \hl{TODO: Need to format equations properly at end}
 \begin{center}
 	\large
 	$C_6 =$
 	$\frac{\theta}{1.5}$ for $1.5 \leq \theta \leq 100$\\
 	$C_6 = 1$ for $\theta < 1.5, \theta > 100$
 \end{center}
 
 Using trigonometry, the divergence angle, $\theta$, for the laser is 
 \begin{center}
 	\large
 	$Tan^{-1}(\frac{r}{300})* 1000$ $[mrad]$
 \end{center}
 
 Following the ANSI Standard \cite{ANSI}, the Rule 1 calculation is 
 \begin{center}
 	\large
 	$R_1 = 5*10^{-3} * C_6$
 \end{center}
 
 The Rule 2 calculation is
 \begin{center}
 	\large
 	$R_2 = 18 (T)^{0.75}$
 \end{center}
 
 The Rule 3 calculation is
 \begin{center}
 	\large
 	$R_3 = R1(T*f)^{0.25}$
 \end{center}
 
 The most restrictive rule defines the MPE 
 \begin{center}
 	\large
 	$MPE = min(R_1, R_2, R_3)$
 \end{center}
 
 The MPE, then, is\\
 {\large $min($}
 \begin{center}
 	\large
 	$ 5*10^{-3} * Tan^{-1}(\frac{.5}{300})* 1000$\\
 	$18 (.25)^{0.75}$ \\
 	$0.00833333 (0.25*40000)^{0.25}$
 \end{center}
 {\large $)$}
 
 This gives 
 \begin{center}
 	\large
 	$MPE = min(0.00833333, 6.36396, 0.0833333) = 0.00833333 [\frac{J}{m^2}]$
 \end{center}
 
 The NOHD is defined as (with $\theta$ in terms of $rad$, not $mrad$)
 \begin{center}
 	\large
 	$ \frac{\sqrt{\frac{4 * P}{\pi * MPE}} - 2w}{\theta}$
 \end{center}
 
 Where P is the power of the beam ($6.25*10^{-8} J$) and $w$ is the waist of the beam, $0.5mm$. This gives an NOHD of 
 \begin{center}
 	\large
 	$ \frac{\sqrt{\frac{4 * 6.25*10^{-8} }{\pi * 0.00833333}} - 2*0.0005}{Tan^{-1}(\frac{.5}{300})} = 1.25 m$
 \end{center}
 
 The team will take precautions to avoid eye contact with the laser within $1.25m$ of the source. If it is absolutely necessary to work with the laser powered on and a person within $1.25m$ of the laser, the person will be required to wear protective eye wear. 
 
 The risk of eye damage is mitigated by the fact that the laser is both visible, and not always powered on. 

\subsubsection{Friendly Target}


\begin{figure} [H]
	\centering
	\includegraphics[scale=0.08]{target_picture.png}
	\caption{Friendly Interrogator Unit\label{fig:threshold}}
\end{figure}


\hspace{5mm}\textbf{Laser Photoreceiver} \label{section:laser-photoreceiver-design-procedure}\\
A photodiode was chosen such that a 5kHz signal could be processed and boosted to register a value between 0 and 3.3V at a maximum distance of $30m$ from the laser source. This couples the photoreceiver requirement with the intensity of the laser diode, which is capped at $5 mW$. 

The inherent limitations of laser power for safety means that the performance is in the hands of the photodiode. The team did not change their design review choice and selected PIN photodiodes as they have a high sensitivity and speed.

The next selection criteria is the material with which the photodiode is made. This includes materials such as Si, InGaAs, and InA. The optimum wavelength is dependent on the material selection.

With photodiodes, Noise-equivalent Power (NEP) is a measure of the incident power required to generate a response signal equal to the noise level of a detector system. Detectivity is the reciprocal of the NEP normalized for the active area of the photodiode.\textsuperscript{\cite{Microphotonics}}. The best photodiode, then, will have the highest detectivity for the visible wavelength.  

\begin{figure} [H]
	\centering
	\includegraphics[scale=0.4]{detectivity-table.png}
	\label{fig:detectivity-table}
	\caption{Specific Detectivity for Photodetector Materials \textsuperscript{\cite{Optical}} \label{fig:detectivity-table}}
\end{figure}

Figure \ref{fig:detectivity-table} illustrates the specific detectivity ranges of photodiodes. The interrogation laser is in the visible range; therefore, the matching photodiode is of type Si. Using this type of photodiode, the detectivity is between $10^{10}$ and $10^{13}$ $\frac{Hz^{\frac{1}{2}}}{W}$. The following calculations will use a conservative value, $10^{12} \frac{Hz^{\frac{1}{2}}}{W}$, as the detectivity. In realty, because the wavelength is less than \SI{1}{\micro\meter}, the detectivity is somewhere between $10^{12}$ and $10^{13}$ $\frac{Hz^{\frac{1}{2}}}{W}$

The equation for NEP from detectivity, $D^*$, and photodiode active area, $A$,  is 
\begin{center}
	{\large $NEP = \frac{\sqrt{A}}{D^*}$}  $[\frac{W}{Hz^{1/2}}]$
\end{center}

The incident irradiance, $E_i$, to cancel noise is
\begin{center}
	{\large $E_i = \frac{NEP * \sqrt{f}}{A} = \frac{\sqrt{Af}}{AD^*} = \frac{f}{D^*\sqrt{A}}$} $[\frac{W}{m^{2}}]$
\end{center}

The NEP measures the incident irradiance to cancel the noise on the photodiode. To register a signal on the MCU, the incident irradiance must be higher than the noise. To be conservative, define the required incident irradiance as 
\begin{center}
	{\large $E_{req} = 2E_i$} $[\frac{W}{m^{2}}]$
\end{center}

Multiplying the area of the laser's spot by the required incident irradiance at the photodiode gives the necessary power. Thus, the radius of the spot in terms of the power of the laser and required incident irradiance at the photodiode is 
\begin{center}
	{\large $r = \sqrt{\frac{P}{\pi E_{req}}}$} $[m]$
\end{center}

Note that the power contained in the laser's spot does not depend on distance from the source, as atmospheric reflection is negligible at $300 m$.

The radius, in terms of the detectivity, frequency, and sensor active area is
\begin{center}
	{\large $r = \sqrt{\frac{PD^*\sqrt{A}}{2 \pi f}}$} $[m]$
\end{center}

The proposal listed $0.8382 m$ as the ideal radius of the laser's spot. Unfortunately, with the $5mW$ red laser, this would require a sensor with a massive active area. The largest sensor the team could find, at a reasonable price, has a $8.53 mm^2$ active area. 

For the $8.53 mm^2$ active area photodiode operating at $635 nm = 4.72441 × 10^{14} Hz$ 
\begin{center}
	$r = 0.326998 m \approx 33 cm$
\end{center}

Refining the proposal requirements, the team has set a new requirement of a minimum $20 cm$ laser spot radius, making the diameter of the beam $0.4m$ at distances of $50m, 150m,$ and $300m$ with optical adjustment. This is well within the limit calculated above.  





\subsubsection{System}

\hspace{5mm}\textbf{R.F. Transmitter/Receiver} \label{section:rf-transmitter-design-procedure} \\


The team purchased the R.F. receiver with the intentions of it being relatively easy to use out of the box and easily programmable with any MCU. However, in reality, the receiver proved to be much more difficult than the team imagined. 

SMD adapters were used to plug the R.F. receiver into a breadboard and soon thereafter, the team learned that breadboards 

The requirement driving both the R.F. transmitter and receiver was the ability to broadcast and receive packets (as a pair) at the maximum distance of the project ($30m$). 
\hl{TODO: Describe some of the governing equations}

\hspace{5mm}\textbf{Voltage/Power Regulation} \label{section:interrogator-voltage-regulation-design-procedure}

It is required that the power stay above 3.3V for a period of 8 hours. Originally, in the design review, a 3.3V boost-converter was going to be utilized along with a single AA battery (1.5V) to supply the entire interrogator unit. This proved to be a difficult implementation to use due to the MSP drawing more power than the team expected at the time of the design review. Also, the PCB mounting of this equipment proved to be difficult for unexperienced solderers. 

For this reason, the team opted to use four AA standard alkaline batteries put through a regulator to drop the voltage supplied to the PCB down to 3.3V. 

In the friendly interrogator, this 3.3V output is fed directly into the MSP and R.F. modules for power.

In the friendly target... \hl{Talk about power for friendly target here}

 
\hspace{5mm}\textbf{Microcontroller} \label{section:system-design-procedure}\\
The microcontroller must have enough ports to service both the R.F. boards and laser/photoreceiver inputs, have enough speed to sample and decide on a packet value received at $5kHz$, and have the ability to count seconds. A vast majority of microcontrollers fit the requirements for this design, as most come with several ports, fast processors, and a built in timer.

In the design review, the team was using a MSP430F2274IDA, which proved to be very difficult to program using the lab-supplied FET programmer. For this reason the team decided to switch to the MSP432P401R(Launchpad model MSPEXP432P401) which contained an onboard debugger. These were much easier to use due to the USB-to-microUSB debugger. The team used these microcontrollers for approximately two weeks until both boards were determined to be defective due to internal shorts (drawing approximately 0.5A standby).

The team then decided to switch back to the MSP430 line but instead decided to use the Launchpad series similar to the MSP432. The MSP430G2553 (Launchpad model MSP-EXP430G2) was ultimately the model the team used during final demonstration. These worked excellent and did not cause any problems during demo. 

Other microcontroller options, including PIC and arduino were considered. Arduino was considered overkill and low difficulty for the project, PIC more difficult to work with due to less documented proprietary systems. If pursuing this project further, the team would have still chosen the MSP430 Launchpad series due to the familiarity of using them for the 8 weeks during this project.


%DESIGN DETAILS
\subsection{Design Details}


\subsubsection{Friendly Interrogator}
The Circuit and PCB evolved a great deal from the design review to simplify the circuit. First, the MSP microcontroller was changed from a surface mount device to a breakout using pinheaders on the PCB. The crystal oscillator circuit was removed and some switches were removed due to their complexity. The 
\hspace{5mm}\textbf{Circuit Schematics} \label{section:interrogator-circuit-schematics-design-details}

Please refer to Figure \ref{fig:interrogator-schematic} for the circuit schematic of the friendly interrogator unit. 

\hspace{5mm}\textbf{PCB} \label{section:interrogator-pcb-design-details}

Please refer to Figure \ref{fig:interrogator-pcb} for the circuit schematic of the friendly interrogator unit. 



\subsubsection{Friendly Target}
\hspace{5mm}\textbf{Circuit Schematics} \label{section:target-circuit-schematics-design-details}
\\ \hl{TODO}

\hspace{5mm}\textbf{PCB} \label{section:target-pcb-design-details}
\\ \hl{TODO}

\hspace{5mm} \textbf{Software} \label{section:target-software-design-details}\\
Incoming transmissions are broken into two parts, where a $1$ corresponds to a high at the transmission source, and a $0$ a low: 
\begin{enumerate}
	\small
	\item \textbf{Preamble} - $10101010$ 
	\item \textbf{Packet} - An 8 digit long value representing a numerical id
\end{enumerate}


Values from the photoreceiver are analog values ranging from $0-3.3V$. Because the ambient light in the room can the photoreceiver to report $0-2V$ when receiving no light from the transmitter, an algorithm was created to decide binary values of a transmission based only on the \textit{difference} between the current analog value and the analog value $200 \mu s$ before. This is similar to Non-Return to Zero Inverted (NRZI) encoding scheme widely used in many applications. This is Algorithm \ref{algo-1} shown in the appendix. Figure 

\begin{figure} [H]
	\centering
	\includegraphics[scale=0.45]{threshold.png}
	\caption{Threshold Noise Difference\label{fig:threshold}}
\end{figure}


Binary values are stored in an array of length 8, since both packets and preambles are 8 bits. Because a preamble can come at any moment, an algorithm was designed both to store binary values in the array, and to always check for a preamble packet. This is Algorithm \ref{algo-2} in the appendix.

Once a preamble has been received, a simple algorithm is used to capture a packet. This is Algorithm \ref{algo-3} in the appendix. After 100 queries of a packet has received 85\% success, an acknowledgement will be sent over R.F.


Values from the photoreceiver are analog values ranging from $0-3.3V$. Because the ambient light in the room can the photoreceiver to report $0-2V$ when receiving no light from the transmitter, an algorithm was created to decide binary values of a transmission based only on the \textit{difference} between the current analog value and the analog value $200 \mu s$ before.


\makeatletter
\def\BState{\State\hskip-\ALG@thistlm}
\makeatother


Binary values are stored in an array of length 8, since both packets and preambles are 8 bits. Because a preamble can come at any moment, an algorithm was designed both to store binary values in the array, and to always check for a preamble packet 



Once a preamble has been received, a simple algorithm is used to capture a packet

\subsubsection{System}

FROM TEMPLATE: Present the detailed design, with diagrams and component values. Show how the design equations were applied. Give equations and diagrams with specific design values and data. Place large data tables in an appendix.  Circuit diagrams that are too large to be readable on a single page should be broken into pieces for presentation.  The full diagram may be included in an appendix.  Use photographs only as necessary and treat them, along with all other graphics except tables, as figures.


%SECTION - VERIFICATION
\section{Verification}

The Requirements and Verification Table is shown in the appendix.


\subsection{Laser Transmitter \& Photoreceiver}
\subsubsection{Distance Requirement}
\begin{table}[htbp]
	\centering
	\begin{tabular}{c|c|c}	% ccccccc indicates 7 center aligned columns
		\toprule	% top separator
		Distance & Ambient & Transmitter On \\
		\midrule
		2 m  & 0.556 V & 1.75 V\\
		5 m & 0.526 V & 1.59 V\\
		15 m & 0.636 V & 1.12 V\\ 
		30 m & 0.562 V & 0.87 V\\
		
		\bottomrule	% bottom separator
	\end{tabular}%
	\caption{Laser Transmitter Distance Test}
	\label{tab:distance-requirement}	% this is the label given to the table that can be referenced using \ref{tab:Exp1Part1_7}
\end{table}%

\subsubsection{Spot Size Requirement}
\begin{table}[htbp]
	\centering
	\begin{tabular}{c|c|c}	% ccccccc indicates 7 center aligned columns
		\toprule	% top separator
		Distance & Smallest Spot Size Radius & Largest Spot Size Radius \\
		\midrule
		2 m  & 2.5mm & 16.25mm\\
		5 m & 3.75mm & 36.5mm\\

		
		\bottomrule	% bottom separator
	\end{tabular}%
	\caption{Laser Transmitter Spot Size Test}
	\label{tab:distance-requirement}	% this is the label given to the table that can be referenced using \ref{tab:Exp1Part1_7}
\end{table}%

\subsubsection{Packet Reception}
\begin{figure} [H]
	\centering
	\includegraphics[scale=0.45]{packet_verification.png}
	\caption{Packet Sent/Received Signal\label{fig:packet-verification}}
\end{figure}

\subsubsection {Counting Packets}
Algorithm \ref{algo-4} was used to count the number of missed packets from the photoreceiver.
The results are shown in Figure \ref{fig:transmitted-received}. The algorithm verified at 5 meters, over 85\% packet reception.

\begin{figure} [H]
	\centering
	\includegraphics[scale=0.45]{transmitted_received.png}
	\caption{Missed Packets Verification\label{fig:transmitted-received}}
\end{figure}

\subsection{R.F. Transmitter \& Receiver}
No fuckin verification.

\subsection{Power Module}
The requirements stated the power module needed to output a steady 3.3V for a period of 8 hours $\pm$ 5\%

\hspace{5mm}\textbf{Friendly Interrogator Consumption} \label{section:interrogator-consumption}
\begin{table}[htbp]
	\centering
	\begin{tabular}{c|c}	% ccccccc indicates 7 center aligned columns
		\toprule	% top separator
		Module & Active Current Consumption \\
		\midrule
		MSP430 & 1.521 mA \\ 
		Linx KH3 R.F. Receiver & 10.1 mA \\
		5 mW Laser & 24.7 mA (max) \\
		PCB Standby & 30.1 mA \\
		\bottomrule	% bottom separator
		\textbf{Total} & 65.62 mA \\
	\end{tabular}%
	\caption{Current Consumption Values for Interrogator Unit}
	\label{tab:table2}	% this is the label given to the table that can be referenced using \ref{tab:Exp1Part1_7}
\end{table}%


\hspace{5mm}\textbf{Friendly Target Consumption} \label{section:ttarget-consumption}
\begin{table}[htbp]
	\centering
	\begin{tabular}{c|c}	% ccccccc indicates 7 center aligned columns
		\toprule	% top separator
		Module & Active Current Consumption \\
		\midrule
		MSP430 & 1.521 mA \\ 
		Linx KH3 R.F. Transmitter & 7.81 mA\\
		Photoreceiver Op Amp & 2.4 mA (max)\\
		MAX680 5V to +/- 10V Converter & 500 $\mu$A \\
		\bottomrule	% bottom separator
		\textbf{Total} & -mA \\
	\end{tabular}%
	\caption{Current Consumption of Target Unit}
	\label{tab:table2}	% this is the label given to the table that can be referenced using \ref{tab:Exp1Part1_7}
\end{table}%

\subsection{System Speed}


%COSTS
\section{Costs}

The labor cost was calculated as follows:

\begin{center}
	Total Cost = Parts + Worker Salary (\$/hour) x 2.5 x Time (Hours) Invested In Project
\end{center}

\begin{figure} [H]
	\centering
	\includegraphics[scale=0.45]{parts-labor-cost.png}
	\caption{Parts and Labor Cost\label{fig:parts-labor-cost}}
\end{figure}



%CONCLUSION
\section{Conclusion}
\hl{TODO}
\subsection{Ethics}
This project has several ethical issues that can be addressed by the IEEE Code of Conduct. Specifically, numbers 1, 2, 3, 5, 6, 7, and 9 are the most important items that pertain to the Infantry I.F.F. System the team is building this semester. 
\begin{enumerate}
	\item to accept responsibility in making decisions consistent with the safety, health, and welfare of the public, and to disclose promptly factors that might endanger the public or the environment;
	\item to avoid real or perceived conflicts of interest whenever possible, and to disclose them to affected parties when they do exist;
	\item to be honest and realistic in stating claims or estimates based on available data;  
	\skipitems{1}
	\item to improve the understanding of technology; its appropriate application, and potential consequences;  
	\item to maintain and improve our technical competence and to undertake technological tasks for others only if qualified by training or experience, or after full disclosure of pertinent limitations;  
	\item to seek, accept, and offer honest criticism of technical work, to acknowledge and correct errors, and to credit properly the contributions of others;  
	\skipitems{1}
	\item to avoid injuring others, their property, reputation, or employment by false or malicious action;  
\end{enumerate}
Number 1 and 9 apply to this project in a significant way due to the handling of lasers and equipment that is to be used for mounting on a weapon. For one, there will be no handling of any weapons this semester whatsoever. The team will use replacement props for demonstration purposes. The laser safety has been verified to be safe up to 1.5 meters and the team will in NO matter use the laser in an unsafe manner. The laser will only be used for project-purposes and will not be used outside of the lab.

Number 2 and 3 are somewhat of the same and the team will ensure to disclose any crucial information the ECE department must know regarding our project, as soon as the team is aware. 

Number 5 is particularly important in this project due to the IFF system being designed for wartime use. This means that the team must understand the consequences if pursued further. There are lives at stake here and this is a very serious matter. 

Number 6 and 7 can be coupled together in an academic integrity statement. There are several other I.F.F. systems developed by other schools and the team is well aware of this. However, the work the team produces this year will be an individual/team pursuit unique to the senior design semester of spring 2016. It will, in no way, be a reproduction of any other I.F.F. system out there. The design approaches maybe similar due to similar constraints/budgetary concerns, however, the work the team produces will be 100\% authentic and absolutely not plagiarized. 
\subsection{Future Work}
FROM TEMPLATE: Bring together, concisely, the conclusions to be drawn. It may be appropriate, depending on the nature of the project, to begin or end with a two‐ or three‐sentence executive summary. The reader needs to be convinced that the design will work. Summarize your accomplishments. If uncertainties remain, they should be pointed out, and alternatives, such as modifying performance specifications, should be spelled out to deal with foreseeable outcomes. Use  words, not equations or diagrams. Devote a section to ethical considerations with reference to the IEEE Code of Ethics and any other applicable code (e.g., the AMA Code of Medical Ethics for certain bioengineering projects).



\clearpage
%REFERENCES
\bibliographystyle{IEEE_ECE}
% include the BibTex file here to build reference
\bibliography{Citations}\addcontentsline{toc}{section}{Reference}
\clearpage
\section*{Appendix}
\pagenumbering{gobble}

\begin{figure} [H]
	\centering
	\includegraphics[scale=0.37]{friendly_interrogator_circuit.png}
	\caption{Friendly Interrogator Circuit Schematic \label{fig:interrogator-schematic}}
\end{figure}

\begin{figure} [H]
	\centering
	\includegraphics[scale=0.37]{interrogator_pcb.png}
	\caption{Friendly Interrogator PCB\label{fig:interrogator-pcb}}
\end{figure}


%\begin{figure} [H]
%	\centering
%	\includegraphics[scale=0.37]{friendly_interrogator_circuit.png}
%	\caption{Friendly Interrogator Circuit Schematic \label{fig:threshold}}
%\end{figure}
\hl{PUT TARGET UNIT SCHEMATIC/PCB HERE}


\makeatletter
\def\BState{\State\hskip-\ALG@thistlm}
\makeatother

\begin{algorithm}[H]
	\caption{GetCurrentBinaryValue}\label{algo-1}
	\begin{algorithmic}[1]
		\State $\textbf{Input}: \textit{currentAnalogValue},\ \textit{lastAnalogValue},\ \textit{lastBinaryValue},\ \textit{THRESHOLD}$\\
		
		\State $\textit{dif} \gets (\textit{currentAnalogValue} - \textit{lastAnalogValue})$\\\\
		
		
		//Was a 0, now a high indicates a 1
		\If {$dif > \textit{THRESHOLD}$} \\
		\quad \Return $1$\\\\
		
		//Was a 1, now a low indicates a 0
		\ElsIf {$dif < -\textit{THRESHOLD}$} \\
		\quad \Return $0$\\\\
		
		//No change from last value
		\Else\\
		\quad \Return \textit{lastBinaryValue}
		\EndIf
	\end{algorithmic}
\end{algorithm}

\begin{algorithm}[H]
	\caption{ReceivePreamble}\label{algo-2}
	\begin{algorithmic}[1]
		\State $\textit{photoBinary} \gets \text{new}\ \textit{Array}(8)$
		\State $\textit{lastBinaryValue} \gets 0$
		\State $\textit{lastAnalogValue} \gets 0$
		\State $currentIndex \gets 0$\\
		
		\State \textbf{every} $200 \mu s$ \textbf{do}\\
		\quad // Store current value\\
		\quad $\textit{currentAnalogValue} \gets \text{analogReadPhotoreceiver}$\\
		\quad $\textit{currentBinaryValue} \gets \text{GetCurrentBinaryValue}(\textit{currentAnalogValue},$\\ 
		\hspace{7.65cm} $\textit{lastAnalogValue},$\\
		\hspace{7.65cm} $\textit{lastBinaryValue})$ \\
		\quad $\text{photoBinary}[\textit{currentIndex}] \gets \textit{currentBinaryValue}$\\
		\quad $\textit{currentIndex} = (\textit{currentIndex} + 1)\ \%\ 8$\\\\
		
		\quad // Check if last 8 received values make 10101010 
		\State \quad $\textit{lastEightValues} = \text{concat}(\text{photoBinary}[\textit{currentIndex}...8],\text{photoBinary}[0..\textit{currentIndex}])$ \\
		\quad \textbf{if} $\textit{lastEightValues} == 10101010$ \textbf{then}\\
		\quad \quad ReceivePacket\\
		
		\State \quad $\textit{lastBinaryValue} \gets \textit{currentBinaryValue}$
		\State \quad $\textit{lastAnalogValue} \gets \textit{currentAnalogValue}$
		\State \textbf{end}		
	\end{algorithmic}
\end{algorithm}

\begin{algorithm}[H]
	\caption{ReceivePacket}\label{algo-3}
	\begin{algorithmic}[1]
		\State $\textit{packet} \gets \text{new}\ \textit{Array}(8)$
		\State $\textit{lastBinaryValue} \gets 0$
		\State $\textit{lastAnalogValue} \gets 0$
		\State $currentIndex \gets 0$\\
		
		\State 8 \textbf{times}, \textbf{every} $200 \mu s$ \textbf{do}\\
		\quad // Store current value\\
		\quad $\textit{currentAnalogValue} \gets \text{analogReadPhotoreceiver}$\\
		\quad $\textit{currentBinaryValue} \gets \text{GetCurrentBinaryValue}(\textit{currentAnalogValue},$\\ 
		\hspace{7.65cm} $\textit{lastAnalogValue},$\\
		\hspace{7.65cm} $\textit{lastBinaryValue})$ \\
		\quad $\text{packet}[\textit{currentIndex}] \gets \textit{currentBinaryValue}$\\
		\quad $\textit{currentIndex}\text{++}$\\\\
		
		\State \quad $\textit{lastBinaryValue} \gets \textit{currentBinaryValue}$
		\State \quad $\textit{lastAnalogValue} \gets \textit{currentAnalogValue}$
		\State \textbf{end}\\\\
		
		\Return \textit{packet}
	\end{algorithmic}
\end{algorithm}

\begin{algorithm}[H]
	\caption{CountMissedPackets}\label{algo-4}
	\begin{algorithmic}[1]
		\State \textbf{Input}: \textit{expectedPacketValue}\\
		\State $\textit{missedPackets} \gets 0$\\
		ReceivePreambleAndPacket // Get first packet to ensure transmission started
		
		\State $100.\textbf{times}\ \textbf{do}$\\
		\quad ReceivePreambleOnce // Receive next 8 bits as preamble, ignore if not a correct preamble \\
		\quad \textbf{if} ReceivePacket $\neq$ \textit{expectedPacketValue} \textbf{then} \\
		\quad \quad \textit{missedPackets}++
		\State \textbf{end}
	\end{algorithmic}
\end{algorithm}
\clearpage

\begin{figure} [H]
	\centering
	\includepdf[page=1, scale=0.85] {Requirements-Verification.pdf}
\end{figure}
\clearpage

\begin{figure} [H]
	\centering
	\includepdf[page=2, scale=0.85] {Requirements-Verification.pdf}
\end{figure}

\end{spacing}
\end{document}

