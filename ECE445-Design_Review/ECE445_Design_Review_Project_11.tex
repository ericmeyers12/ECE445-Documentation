%%%%%%%%%%%%%%%%%%%%%%%%%%%%%%%%%%%%%%%%%%%%%%%%%%%%%%%%%%%%%
% ECE 445 SENIOR DESIGN TEMPLATE
%%%%%%%%%%%%%%%%%%%%%%%%%%%%%%%%%%%%%%%%%%%%%%%%%%%%%%%%%%%%%
\documentclass[letterpaper,10pt]{article}

%%%%%%%%%%%%%%%%%%%%%%%%%%%%%%%%%%%%%%%%%%%%%%%%%%%%%%%%%%%%%
% The preamble starts here.
% You can add other packages that you want to use by using
% \usepackage command in the preamble.
% However, DO NOT change the settings that are already placed
% below unless you really know what you are doing.
%%%%%%%%%%%%%%%%%%%%%%%%%%%%%%%%%%%%%%%%%%%%%%%%%%%%%%%%%%%%%

% some commonly used packages
\usepackage{siunitx}
\usepackage{graphicx}
\usepackage{color,soul}
\usepackage{amsmath}
\usepackage{amsthm}
\usepackage{amsfonts}
\usepackage{setspace}
\usepackage{longtable}
\usepackage{url}
\usepackage{pdfpages}
\usepackage{float}
\usepackage{rotating}
\usepackage{caption}
\usepackage{booktabs}  % professional-looking tables
\usepackage{multicol} %used for getting multicolumn without page-break
\usepackage{multirow}	% multi-row tables
\usepackage{array}		% define column format of a table
\usepackage[colorlinks=true,linkcolor=black,citecolor=black]{hyperref}
\usepackage[top=1.1in, bottom=1.1in, left=1.1in, right=1.1in]{geometry}% set the page margins to 1 inch

% use the fancyhdr package to maintain the format of the page numbers,
% which is useful when the text color is changed
\usepackage{fancyhdr}
\fancyhf{}
\renewcommand{\headrulewidth}{1pt}
\renewcommand{\footrulewidth}{0pt}
\fancyfoot[C]{\textcolor{black}{\thepage}}
\fancyhead[L]{\includegraphics[width=2cm]{University-of-Illinois-logo.jpg}}
\fancyhead[R]{\small{Infantry I.F.F. Design Review - Meyers \& Prince}}

% paralist provides extended list environments
\usepackage{paralist}
\setlength{\plitemsep}{0pt}

% define the color for section and subsection titles
\usepackage{xcolor}
\definecolor{titlecolor}{RGB}{31,73,125}
\definecolor{subtitlecolor}{RGB}{79,129,189}

% change the style of the section and subsection titles
\usepackage{titlesec}
\titleformat{\section}{\color{titlecolor}\Large\bf}{\color{titlecolor}\thesection}{0.8em}{}
\titleformat{\subsection}{\color{subtitlecolor}\large\bf}{\color{subtitlecolor}\thesubsection}{1em}{}
\titleformat{\subsubsection}{\color{subtitlecolor}\normalsize\bf}{\color{subtitlecolor}\thesubsubsection}{1.2em}{}
\titlespacing{\section}{0pt}{0em}{0em}
\titlespacing{\subsection}{6pt}{0em}{0em}
\titlespacing{\subsubsection}{12pt}{0em}{0em}



% change the style of the table of contents
\usepackage{titletoc}
\titlecontents{section}[1.5em]{}{\contentslabel{1.5em}}{\hspace*{-1.5em}}{\titlerule*[0.5pc]{.}\contentspage}
\titlecontents{subsection}[3em]{}{\contentslabel{2.1em}}{\hspace*{-2.1em}}{\titlerule*[0.5pc]{.}\contentspage}
\titlecontents{subsubsection}[5.1em]{}{\contentslabel{2.7em}}{\hspace*{-2.7em}}{\titlerule*[0.5pc]{.}\contentspage}

% command for centering texts in a fixed width table cell
\newcommand{\centpcol}{\leftskip\fill \rightskip\fill}

% command for setting the style of the appendix titles
\newcommand{\setappenstyle}{
	\titleformat{\section}{\color{titlecolor}\Large\bf}{\color{titlecolor}Appendix \Alph{section}}{0.8em}{}
	\titlecontents{section}[0em]{}{Appendix \thecontentslabel \hspace{1em}}{}{\titlerule*[0.5pc]{.}\contentspage}
}

\makeatletter
\newcommand{\skipitems}[1]{%
	\addtocounter{\@enumctr}{#1}%
}

% define the style of the title of the paper
\newcommand{\thetitle}[1]{\title{\begin{huge}{\bf #1}\end{huge} \color{subtitlecolor}\rule[25pt]{\textwidth}{1pt}}}

% define the style of the author
\newcommand{\theauthor}[3]{
	\author{\vspace{.4in}\\
	\textcolor{black}{By}\\
	#1
	\vspace{1in}\\
	\textcolor{black}{ECE 445 Design Review -} #2\\
	\textcolor{black}{TA:} #3
	\vspace{1in}}
}

% define the style of figure's caption
\newcommand{\figcap}[1]{
	\captionsetup{format=plain,font={small,color=subtitlecolor,singlespacing},margin={0pt,0pt}}
	\caption{\textcolor{subtitlecolor}{#1}}
	\vspace{-5pt}
}

% define the style of table's caption
\newcommand{\tablecap}[1]{
	\captionsetup{format=plain,font={bf,normalsize,singlespacing,color=black},margin={0pt,0pt}}
	\caption{\textcolor{black}{#1}}
	\vspace{-5pt}
}


\newcommand{\buildtoc}{
	\clearpage
	\singlespacing
	\tableofcontents
	\onehalfspacing
}

% set indentations and the space between paragraghs
\setlength{\parindent}{0pt}
\setlength{\parskip}{8pt}

\setcounter{secnumdepth}{4}

\titleformat{\paragraph}
{\normalfont\small\bfseries\color{subtitlecolor}}{\theparagraph}{1em}{}
\titlespacing*{\paragraph}
{18pt}{3.25ex plus 1ex minus .2ex}{1.5ex plus .2ex}

%%%%%%%%%%%%%%%%%%%%%%%%%%%%%%%%%%%%%%%%%%%%%%%%%%%%%%%%%%%%%
% PREAMBLE ENDS HERE, DOCUMENT STARTS BELOW
%%%%%%%%%%%%%%%%%%%%%%%%%%%%%%%%%%%%%%%%%%%%%%%%%%%%%%%%%%%%%

\begin{document}

% don't change these
\pagestyle{empty}
\doublespacing

% put the title of your project here. DO NOT include the brackets.
\thetitle{{I.F.F. (Identification Friend or Foe) System}}

% put your names here. seperate by \\. DO NOT include the brackets.
\theauthor{
	{Eric Meyers (emeyer7)}\\
	{Noah Prince (nprince2)}\\
}
{ % put the semester info here. DO NOT include the brackets.
	{Spring 2016}
}
{ % put your TA's name here. DO NOT include the brackets.
	{Braedon Salz}
}

% put the date and project number here. DO NOT include the brackets.
\date{
{March 2nd, 2016}\\
Project No. 11
\clearpage
}

% don't change these
\maketitle
\pagestyle{fancy}
\begin{spacing}{1.15}


% build the table of contents. 
\color{black}
\buildtoc
\pagenumbering{gobble}
\clearpage
\setcounter{page}{1}
\pagenumbering{arabic}

%SECTION - Introduction
\section{Introduction}
\subsection{Statement of Purpose}
There have been several friendly fire incidents in recorded military history, accounting for an estimated 2\% to 20\% of all casualties in battle\textsuperscript{\cite{USArmy}}. Using attire to identify friend vs enemy is problematic in situations when both sides are clad in the same camouflage pattern, or are obscured by obstacles.

The purpose of this project is to create a system that quickly and accurately identifies friendly targets among military personnel on foot. Similar systems exist for aircraft, however not many exist for infantry.

The idea is to develop a two-way communication system so that when a soldier aims their weapon in the direction of a friendly target, they will receive notification through an LED that the target is, indeed, friendly and not an enemy. Throughout this document the infantry unit with the weapon will be referred to the "friendly interrogator" and the target will  be referred to as the "friendly target". 

\subsection{Objectives}
\subsubsection{Goals and Benefits}
\begin{itemize}
	\item Reduce the number of friendly fire accidents during combat \textsuperscript{\cite{Garrison}}.
	\item Reduce the number of misfire accidents during combat\textsuperscript{\cite{Garrison}}.
	\item Notify friendly personnel location of particular friendly target when aiming in their direction.
	\item Other applications including but not limited to:
	\begin{itemize}
		\item Paintball or Airsoft
		\item Arcade laser tag and other various recreation sports
	\end{itemize}
\end{itemize}
\subsubsection{Functions and Features}
\begin{itemize}
	\item Laser diode on friendly interrogator to transmit unique I.D. of friendly interrogator.
	\item Photodiodes on friendly target to detect unique I.D. and verify it is a valid signal.
	\item R.F. Transmitter on friendly target to send acknowledgement back to interrogator.
	\item R.F. Receiver on friendly interrogator to verify that the target is friendly.
	\item LED on friendly interrogator to indicate to the operator the status of the target.
	\item Quick response time (human reaction time is 190 ms\textsuperscript{\cite{Reaction_Times}}).
\end{itemize}
\clearpage

%SECTION - Design
\section{Design}
%Subsection - Block Diagrams
\subsection{Block Diagrams and Descriptions}
\subsubsection{System Overview}
The following figure represents the system as a whole, including both the friendly interrogator unit and the friendly target unit. Both units will be expanded upon in further detail below.
\begin{figure} [H]
	\centering
	\includegraphics[scale=0.45]{System_Block_Diagram.png}
	\caption{System Block Diagram\label{fig:system-block-diagram}}
\end{figure}

%DESIGN - FRIENDLY INTERROGATOR UNIT
\subsubsection{Friendly Interrogator Unit} \label{section-friendly-interrogator-design}
The following diagram shows the friendly interrogator unit \textit{only}. The interconnections in red represent power, interconnections in blue represent input to a block and interconnections in green represent output to a block. These inputs and outputs are described below under each block description.
\begin{figure} [H]
\centering
\includegraphics[scale=0.5]{Friendly_Interrogator_Block_Diagram.png}
\caption{Block Diagram of Friendly Interrogator Unit\label{fig:friendly-interrogator-block}}
\end{figure}

\normalsize\textbf{Power Module} \\
Power-In, ,  N/A \\
Power-Out: MSP430 Microcontroller, 5mW Laser Transmitter, R.F. Receiver/Decoder, LED Indicator \\
Input(s): N/A\\
Output(s): N/A

The power module will consist of a single standard alkaline AA battery (no specific brand/part name is necessary) which will feed into a Skyworks AAT1217 DC-DC step-up voltage converter. This will step the voltage up to 3.3V with a maximum current output of 100 mA which is sufficient enough to power the MCU, laser transmitter, and the R.F. receiver/decoder. Please refer to Section \ref{section-simulations-calculations} for these calculations regarding power delivery to this unit.

As noted on the block diagram, in between the power module and the R.F. receiver/decoder and laser transmitter block, there is a switch. This is meant to regulate the power consumption of these two devices and to give the operator the choice to query their target instead of the laser constantly querying and drawing current. This decreases power consumption tremendously.

The circuit schematic to operate the DC-DC step-up converter properly is shown in Figure \ref{fig:voltage-converter-schematic} below in Section \ref{section-circuit-schematics}.


The battery will be mounted to the PCB a standard AA through-hole PCB battery mount shown in Figure \ref{fig:pcb-battery-mount}. However, this will be omitted from the circuit schematic for simplicity purposes. This part will be included when constructing the PCB.
\begin{figure} [H]
	\centering
	\includegraphics[scale=0.4]{PCB_Battery_Mount.png}
	\caption{PCB Battery Mount\label{fig:pcb-battery-mount}}
\end{figure}

\normalsize\textbf{MSP430 Microcontroller} \\
Power-In: 3.3V (from Voltage Converter Output) \\
Power-Out: N/A \\
Input(s): R.F. Receiver/Decoder Data-Out, 8-Pin DIP Switch,  Real Time Clock (32.768kHz Crystal), Reset Clock/Sync Button \\
Output(s): LED, 5mW 650nm Laser Transmitter

The team chose to work with an T.I. MSP430F2274 Microcontroller Unit due to its compiler simplicity, the low-power consumption, and the amount of General Purpose Input/Output (GPIO) pins available on board.

The board requires a 3.3V power supply to both the DV\textsubscript{cc} (pin 2) and AV\textsubscript{cc} (pin 16) which is why the voltage regulator is necessary as stated in the previous section. 

The inputs to the MCU will be the R.F. receiver/decoder signal (9 GPIO pins), the 8-pin DIP switch (8 GPIO pins), the real time clock (2 GPIO pins), and the reset clock/sync button (1 GPIO pin). The inputs will consume a total of 20 GPIO pins. The outputs of the MCU consist of the LED indicator (1 GPIO pin) and the laser transmitter (1 GPIO pin). This brings the total number of GPIO pins to 22. The power will consume 4 pins on the MCU and this brings the total amount of pins being consumed to 28 out of 38 on-board. 

This is summarized in Table \ref{tab:msp430-pin-assignments} .Each pin/label is listed with the description of the input or output. The red rows indicate that the pin is a power/ground line, the blue rows indicate that the pin is an input \textit{to} the MCU and the green rows indicate that the pin is an output \textit{from} the MCU.

\begin{table} [H]
	\centering
	\includegraphics[scale=0.37]{MSP430_Pin_Assignments_Interrogator.png}
	\caption{Pin Layout Table\label{tab:msp430-pin-assignments}}
\end{table}

The R.F. receiver/decoder signal will be the acknowledgment sent from the friendly target. These outputs from the receiver/decoder will be fed into pins 20 - 27 on the MCU. This corresponds to 8 bits of data the MCU will be receiving from the friendly interrogator.

This unique I.D. provided by the DIP switch (pins 31-38) will be used in conjunction with the output pin to the laser transmitter (pin 17). The data will be sent over optical transmission by pulsing the laser at a set frequency to send a bit-stream of the I.D. to the friendly target. This will be explained in more detail in Section \ref{section-software}.

The reset clock/sync switch is meant to send a signal to the real time clock (explained below) to reset the clock.

\normalsize\textbf{Real Time Clock}\\
Power-In: N/A \\
Power-Out: N/A \\
Input(s): N/A \\
Output(s): MSP430 Microcontroller

The Real Time Clock (RTC) is necessary for the verification of the acknowledgment signal sent by the R.F. transmitter/encoder on board the friendly target unit. It will operate using a ECS-3x8 32.768 kHz crystal oscillator (as recommended by T.I. \textsuperscript{\cite{RTC-Implementation}}) with an accuracy of $\pm$ 20 PPM \textsuperscript{\cite{Crystal}} (Parts Per Million - deviates between 32.7673 kHz and 32.7687 kHz). It will be used in conjunction with the real time clock library provided by T.I. \textsuperscript{\cite{RTC-Library}}.

Essentially, the concept is that interrupts will be generated at 1-second intervals and this will be used in conjunction with software to keep track of the given second, minute, hour, and day it currently is. The reset clock/sync button explained in the MCU section will be used to reset this count in software and turn the clock back to 0 for synchronization purposes. This will be used for debugging and prototyping purposes only and on a final product used in the military, this type of synchronization could not be used.

\normalsize\textbf{Laser Diode/Transmitter}\\
Power-In: 3.3V (from Voltage Converter Output) \\
Power-Out: N/A \\
Input(s): MSP430 Microcontroller
Output(s):  5mW 650nm laser signal containing unique I.D. of interrogator 

For safety reasons, the maximum allowable power for the laser diode is $5mW$; which registers as a Class IIIa laser. The laser diode must also fall in the visible range, so that it will trigger a person's blinking reflex before eye damage occurs. Specifically, the team will use a red ($650nm$) laser. See Section \ref{section-safety-ethics} for more on safety of the laser. 

The team will use a 5 mW 650 nm TTL laser transmitter to transmit the unique I.D. (as specified by the 8-pin DIP switch) to the friendly target. This laser will operate on 3.3V at 25mA so a 130 $\Omega$  resistor is necessary to drop the current being supplied to the diode down to this threshold. 

To save cost and time, the team will be purchasing an adjustable focus laser. This laser will allow for optical adjustments to achieve a beam diameter of $\approx$ 1 meter at all required distances. The team will ensure the purchased laser meets these requirements, and will adjust the lens if necessary. The team will also create an operator's manual so that the operator of the friendly interrogator unit will know how much to adjust the lens by to achieve a proper spot size at a distance.

\normalsize\textbf{R.F. Receiver/Decoder} \\
Power-In: 3.3V (from Voltage Converter Output) \\
Power-Out: N/A \\
Input(s): 8-Pin DIP Switch\\
Output(s): MSP430 Microcontroller

A Linx 315 MHz KH3 Series R.F. Receiver/Decoder will be used for this project along with a Linx 315-SP Splatch PCB mounted antenna. Some important values that were used in the selection process of this part are listed in Table \ref{tab:rf-receiver-important-values}\textsuperscript{\cite{Linx-Receiver}\cite{Linx-Antenna}}:

\begin{table}[htbp]
	\centering
	\begin{tabular}{c|c}	% ccccccc indicates 7 center aligned columns
		\toprule	% top separator
		Parameter & Typical Value \\
		\midrule
		Operating Voltage & 3.3V\\
	 	Supply Current & 5.9mA\\
		Receiver Frequency & 315 MHz \\ 
		Receiver Sensitivity & -116 dB \\
		R.F. Input Impedance & 50 $\Omega$ \\
		Datarate & 100 bps - 10,000 bps  \\
		Receiver Turn-On Time & 7.0 ms  \\
		\bottomrule	% bottom separator
	\end{tabular}%
	\caption{Linx 315 MHz KH3 R.F. Receiver}
	\label{tab:rf-receiver-important-values}	% this is the label given to the table that can be referenced using \ref{tab:Exp1Part1_7}
\end{table}%

Important values to note are the R.F. input impedance, the receiver frequency, and the receiver sensitivity. The input impedence is stating it requires the entire R.F. receiver system to be matched at 50 $\Omega$. This requires the trace on the PCB from the receiver to the antenna to also be at a 50 $\Omega$ impedence. The frequency and sensitivity affect the range of the receiver/transmitter pair and this calculation along with the PCB trace-width calculation can be found in Section \ref{section-simulations-calculations}.

The antenna is a basic quarter-wave monopole producing an output gain of +2.15 dB.

The decoder provides very important functionality and simplicity to the system. There are a total of 10 address lines on the decoder that must match up to the cooresponding transmitting encoder. The lines do not output data through the data-out lines if these address lines do not match up. For example if the transmitter/encoder's address lines are set to 0000000000, then the receiver/decoder's address lines must also be set to 0000000000. The team decided to make use of these lines and wire them up to a 8-pin DIP switch so that the operator can choose their unique interrogator I.D. Therefore only 8 out of the 10 address lines on the encoder will be used and the top two most significant bits will be grounded (lines A9 and A10).

The process behind the R.F. receiver and processing of the transmitted data will be explained in the Section \ref{section-software}.

%DESIGN - FRIENDLY TARGET UNIT
\subsubsection{Friendly Target Unit}

\begin{figure} [H]
	\centering
	\includegraphics[scale=0.50]{Friendly_Target_Block_Diagram.png}
	\caption{Block Diagram of Friendly Target System\label{fig:friendly-target-block}}
\end{figure}

\normalsize\textbf{Power Module} \\
Power-In: N/A\\
Power-Out: MSP430 Microcontroller (3.3V), R.F. Transmitter/Encoder (3.3V), and Photoreceiver (+/- 10V)\\
Input(s): N/A\\
Output(s): N/A

The power module on board will utilize the same DC-DC Step-Up Voltage Converter as the friendly interrogator unit. However, the friendly target unit's power module must produce both an output voltage of 3.3V for the MCU and 10V for the operational amplifier. This will be accomplished by using one AAT1217 step-up converter to step the voltage up to 3.3V and another to step it up to 5V. This 5V output will then be cascaded into a Maxim MAX680 5V-to-10V step-up converter. Detailed power calculations can be found in Section \ref{section-simulations-calculations}.

\normalsize\textbf{MSP430 Microcontroller} \\
Power-In: Power Module (3.3V)\\
Power-Out: N/A \\
Input(s): Photoreceiver, Real Time Clock\\
Output(s): R.F. Transmitter/Encoder

The MSP430 Microcontroller on the friendly target unit will receive analog input from the 4 photoreceiver units which will then be converted into a digital value using the on board Analog-to-Digital Converters(ADC). The microcontroller will also utilize the R.F. encoder/transmitter (explained below) to broadcast an acknowledge signal back to the friendly interrogator unit.

The MCU will take in the outputs of these photodiodes and determine the I.D. of the interrogator. This I.D. will then be output via pins 31 through 38 to the R.F. transmitter/encoder address lines (this will ensure the R.F. transmitter/encoder address pins will be the same as the receiver end and a transmission will make it through). The software behind this will be described in Section \ref{section-software}.

Once the MCU receives a valid query, it will generate a passphrase based on a common clock and a common passphrase that will then be sent as an output (pins 11-14 and 25-28) to the R.F. transmitter/encoder. This will be the acknowledgment that the interrogator must receive in order for the target to be identified as friendly. This passphrase that is generated will be explained in Section \ref{section-software}.

Figure \ref{tab:msp430-pin-assignments-target} is a list of the active pins on the Friendly Target MCU. There is a total 28/38 GPIO pins used on this MSP430.

\begin{table} [H]
	\centering
		\includegraphics[scale=0.52]{MSP430_Pin_Assignments-Target.png}
	\caption{Pin Layout Table\label{tab:msp430-pin-assignments-target}}
\end{table}

\normalsize\textbf{Real Time Clock} \\
Power-In: N/A\\
Power-Out: N/A\\
Input(s): N/A\\
Output(s): MSP430 Microcontroller

The Real Time Clock on board the Friendly Target Unit will be the same as the Friendly Interrogator Unit. Please reference that section to get all details pertaining to the Real Time Clock.

\normalsize\textbf{Photoreceiver}\\
Power-In: N/A \\
Power-Out: N/A \\
Input(s): Laser Transmitter Signal sent from Friendly Interrogator\\
Output(s): MSP430 Microcontroller

A network of four photodiodes mounted on the friendly target will report incoming laser signals to the MCU. The photodiode signals will be boosted via an operational amplifier, and passed through a low-pass filter. The $40kHz$ signal that passes through the filter will be sampled and processed by the MCU. 

For a detailed analysis of the choice of the photodiodes, see Section \ref{section-tolerance-analysis}

\normalsize\textbf{R.F. Transmitter/Encoder}\\
Power-In: Power Module (3.3V)\\
Power-Out: N/A\\ 
Input(s): MSP430 Microcontroller (Data and Address Pins)\\ 
Output(s): 315 MHz R.F. Acknowledgment Signal

\begin{table}[htbp]
	\centering
	\begin{tabular}{c|c}	% ccccccc indicates 7 center aligned columns
		\toprule	% top separator
		Parameter & Typical Value \\
		\midrule
		Operating Voltage & 3.3V\\
		Supply Current & 2.7mA\\
		Transmitter Frequency & 315 MHz \\ 
		Transmitter Power & -1 dBm \\
		Antenna Gain & 2.15dB  \\
		R.F. Input Impedance & 50 $\Omega$ \\
		Datarate & 100 bps - 10,000 bps  \\
		Receiver Turn-On Time & 7.0 ms  \\
		\bottomrule	% bottom separator
	\end{tabular}%
	\caption{Linx 315 MHz KH3 R.F. Transmitter/Encoder}
	\label{tab:rf-transmitter-important-values}	% this is the label given to the table that can be referenced using \ref{tab:Exp1Part1_7}
\end{table}%

The team decided to work with a Linx 315MHz KH3 R.F. Transmitter/Encoder Module that outputs a total transmitting power of -1dBm. This will be more than sufficient enough to satisfy the requirements of 300 meters (as shown in Section \ref{section-simulations-calculations}). This will be used in conjunction with a quarter-wave monopole antenna (Splatch-315-SP PCB mounted antenna) that provides an antenna gain of 2.15 dB. These values can be summarized in the above table\textsuperscript{\cite{Linx-Transmitter}\cite{Linx-Antenna}}.

As stated before, the address lines and data lines on this transmitter/encoder will be wired up to the MCU so that the proper acknowledgement can be sent to the interrogator.

A similar approach to the friendly interrogator unit will be taken in order to match the R.F. input impedance. The PCB trace-width calculation can be found in Section \ref{section-simulations-calculations}.



%Subsection - Circuit Schematics
\subsection{Circuit Schematics} \label{section-circuit-schematics}

\subsubsection{Friendly Interrogator Unit}
The circuit schematic for the friendly interrogator unit is broken into 4 primary modules for schematic viewing: the AAT1217 DC-DC step-up converter, the crystal oscillator, the MCU/laser transmitter/LED indicator/power switches, and the R.F. receiver/decoder circuits. These are displayed below.
\begin{figure} [H]
	\centering
	\includegraphics[scale=0.35]{Voltage_Converter_Schematic.png}
	\caption{AAT1217 Circuit Schematic\label{fig:voltage-converter-schematic}}
\end{figure}

\begin{figure} [H]
	\centering
	\includegraphics[scale=0.35]{Crystal_Oscillator_Schematic.png}
	\caption{Crystal Oscillator Real Time Clock Circuit Schematic\label{fig:crystal-oscillator-schematic}}
\end{figure}

\begin{figure} [H]
	\centering
	\includegraphics[scale=0.43]{MCU_Laser_Schematic.png}
	\caption{MCU and Laser Transmitter Circuit Schematic\label{fig:mcu-laser-schematic}}
\end{figure}

\begin{figure} [H]
	\centering
	\includegraphics[scale=0.5]{RF_Receiver_Decoder_Schematic.png}
	\caption{RF Receiver/Decoder and 8-Pin DIP Switch Schematic\label{fig:rf-receiver-decoder-schematic}}
\end{figure}


\subsubsection{Friendly Target Unit}
The circuit schematic for the friendly target unit utilizes the same AAT1217 schematic presented in Figure \ref{fig:voltage-converter-schematic}. The remaining schematics for the friendly target unit include the Op Amp Power Circuit, which steps up the supply 1.5V battery to +/- 10V to be used by the photoreceiver unit's op amp; the photoreceiver unit, which steps up and low-pass filters the signal from four photodiodes; the R.F. Transmitter, which broadcasts an acknowledge signal upon receipt of a valid laser transmission on the photoreceiver; and the MCU, which processes the four photoreciever outputs, creates an acknowledge signal, and broadcasts it over the R.F. transmitter. These are displayed below:

\begin{figure} [H]
	\centering
	\includegraphics[scale=0.5]{MCU-Target.png}
	\caption{MCU for the Friendly Target Unit  Schematic\label{fig:mcu-target-schematic}}
\end{figure}

\begin{figure} [H]
	\centering
	\includegraphics[scale=0.5]{Op-Amp-Power.png}
	\caption{1.5V to  +/- 10V Power Converter for Photoreceiver Op Amp Schematic\label{fig:op-amp-power-schematic}}
\end{figure}

\begin{figure} [H]
	\centering
	\includegraphics[scale=0.55]{Photoreceiver.png}
	\caption{Photoreceiver Unit Schematic\label{fig:photoreceiver-schematic}}
\end{figure}

\begin{figure} [H]
	\centering
	\includegraphics[scale=0.5]{RFTransmitter.png}
	\caption{R.F. Transmitter Schematic\label{fig:photoreceiver-schematic}}
\end{figure}

\subsection{Software Flowcharts / Functionality} \label{section-software}
\subsubsection{System Flow}
This section is to explain the flow of events in the system as a whole as well as each the friendly interrogator subsystem and the friendly target subsystem. The below diagram is a flowchart representing the events that occur to identify a target as friendly.

\begin{figure} [H]
	\centering
	\includegraphics[scale=0.55]{Functionality_Flowchart.pdf}
	\caption{Flowchart for Functionality\label{fig:circuit-schematic}}
\end{figure}

The left side of this diagram are all events that occur within the friendly interrogator unit, and the right side represents all of the events that occur on the friendly target side. This flow diagram also assumes that both the interrogator operator and the friendly target operator have powered on their respective units.

\subsubsection{Synchronizing the RTCs}
Both the Friendly Interrogator Unit and Friendly Target Unit must have a shared clock. For the scope of this project, both units will have a "sync" button. The sync button will reset the clock register to 0 on the unit. Units can then synchronize clocks by pressing the "sync" button at the same time. Note that since the tolerance for clock skew is 10 seconds, the sync button should meet this requirement. 

\subsubsection{Encryption and Message Sending Protocol}
The encryption on a system like this is very important to take into consideration and for that reason the team decided to use a passphrase generated both off a common passphrase that is shared in between all friendly interrogator/friendly units and a common clock. The common clock will be provided via the Real Time Clock and this will allow the MCU to have registers containing the current value of seconds/minutes/days/hours/etc. 

The algorithm is relatively straight forward to understand. A timer will be setup (Timer\_A3) to generate interrupts every 10 seconds. This will be to update the contents within the RTC register (R6) every 10 seconds and to have a $\pm$5 second differential to validate the acknowledgement signal. The RTC register will contain 16 bits for counting up on the seconds since the last reset. So it will not be a "true" RTC, however you will have the ability to access the seconds eclipsed since pressing the "sync" button at any given time and this result will be valid for 10 seconds.

The interrogator and target unit must be synced initially before active use. This is so that the two clocks have minimal skew. 
\begin{figure} [H]
	\centering
	\includegraphics[scale=0.45]{Encryption_Diagram.png}
	\caption{Flowchart for Functionality\label{fig:circuit-schematic}}
\end{figure}


\subsubsection{Friendly Interrogator Software Flow}
The software on the MCU on the friendly interrogator unit will follow a very simple flow utilizing basic concepts of microcontrollers including timer, interrupts, i/o and basic logic and arithmetic. The MSP430F2274 contains 16 registers of which 4 are protected and 12 are general purpose registers:
\begin{itemize}
	\item R0 - program counter
	\item R1 - stack pointer 
	\item R2 - status register
	\item R3 - constant generator
	\item R4 - R15 - general purpose registers
\end{itemize}
Using these 12 general purpose registers the team can accurately identify a target as friendly. The flow diagram in Figure \ref*{fig:friendly-interrogator-software-flowchart} in the Appendix depicts the series of processes and events that take place within the MCU in order for the target to be identified as friendly. 

\subsubsection{Friendly Target Software Flow}
The software on the MCU on the friendly target unit will follow a very simple flow utilizing basic concepts of microcontrollers including timer, interrupts, i/o and basic logic and arithmetic. The MSP430F2274 contains 16 registers of which 4 are protected and 12 are general purpose registers:
\begin{itemize}
	\item R0 - program counter
	\item R1 - stack pointer 
	\item R2 - status register
	\item R3 - constant generator
	\item R4 - R15 - general purpose registers
\end{itemize}
Using these 12 general purpose registers the team can accurately identify a target as friendly. The flow diagram in Figure \ref*{fig:friendly-target-software-flowchart} in the Appendix depicts the series of processes and events that take place within the MCU in order for the friendly target unit to broadcast an acknowledge signal. 

\subsection{Numerical Analysis and Simulations/Plots} \label{section-simulations-calculations}
\subsubsection{Calculations}
\normalsize\textbf{Power Module (Friendly Interrogator Unit)} \\
The following section is intended to backup the design choices made for the power module on the friendly interrogator first shown in Section \ref{section-friendly-interrogator-design}.

The team placed a strict requirement (shown in Section \ref{section-requirements-verification}) regarding the operation time of the friendly interrogator unit (at 8 hours of operation time $\pm$ 5\%).

In order to select parts that satisfied this requirement, the active current consumption on the entire unit must first be calculated. The main power consumption modules on board the friendly interrogator unit are the MSP430F2274, the 5mW 635nm TTL laser transmitter, and the Linx KH3 R.F. receiver. These values were received from each of the respective datasheets. The following table displays the active current consumption of each unit\textsuperscript{\cite{MSP430F2274} \cite{Linx-Receiver} \cite{Laser}}. 


\begin{table}[htbp]
	\centering
	\begin{tabular}{c|c|c}	% ccccccc indicates 7 center aligned columns
		\toprule	% top separator
		Module & Active Current Consumption & Standby Current Consumption\\
		\midrule
		MSP430 & 270 $\mu$A & 0.1 - 0.7 $\mu$A\\ 
		Linx KH3 R.F. Receiver & 5.9 mA & 0 mA\\
		5 mW Laser & 25 mA (max) & 0 mA \\
	\bottomrule	% bottom separator
	\end{tabular}%
	\caption{Notable Datasheet Values for Linx 315 MHz LR R.F. Receiver}
	\label{tab:table2}	% this is the label given to the table that can be referenced using \ref{tab:Exp1Part1_7}
\end{table}%

Because the R.F. receiver and the 5mW laser will only be powered when the operator designates, the standby current consumption of these units will be 0.

Therefore, with this information, the maximum possible active current consumption will be:
\begin{center}{Active Consumption I\textsubscript{Total} = $270 \mu A + 25 mA + 5.9 mA = 31.17 mA $}\end{center}

Assuming the team uses a standard Alkaline AA 1.5V Battery, these typically produce anywhere from 1800mAh to 2500 mAh \textsuperscript{\cite{Battery}} Therefore, the average of these two values will be used as the capacity of the battery: 2150 mAh. Since all of the components being used requires 3.3V, this battery must be fed into a voltage step-up converter as stated previously. The team is using the AAT1217 step-up converter this boosts the voltage up from 1.5V to 3.3V with a ~75\% efficiency \textsuperscript{\cite{AAT1217}}.

Since Energy = Power * time, we can use a ratio of the energy produced per hour of the standard alkaline battery to the output voltage of the converter. This calculation can be shown below:



\begin{center} 
	$\textrm{Total Energy} = C_\textrm{battery} * V_\textrm{battery} = 2150 mAh * 1.5V = 3225 mWh$
	
	$\textrm{Total Active Power} = \frac{3.3 V * (0.270 + 5.9 + 25) mA}{.75} = 137.15mW$ 
	
	$\textrm{Active Use Time} = \frac{\textrm{Total Energy}}{\textrm{Total Active Power}} \approx \textrm{\textbf{24 hr active use}}$
	
	$\textrm{Total Standby Power} = \frac{3.3 V * (0.270) mA}{.75} = 11.88 mW$
	
	$\textrm{Standby Use Time} = \frac{\textrm{Total Energy}}{\textrm{Total Passive Power}} \approx \textrm{\textbf{271 hr  standby use}}$
\end{center}

This result shows that a single standard alkaline AA 1.5V disposable battery will be more than sufficient enough to satisfy the requirements of 8 hours of active use time.

\normalsize\textbf{Antenna-to-Receiver and Antenna-to-Transmitter PCB Impedance Matching} \\
The input impedance of both the Linx R.F. receiver, transmitter and antenna are all 50 $\Omega$. Therefore, in order to match this impedance on the line that goes from the receiver/transmitter chip to the antenna chip, the width must be calculated on the PCB trace. 

Figure \ref{fig:pcb-trace} shows all variables that affect the impedence of a PCB trace. 

\begin{center}
	$T $ = trace thickness (in mils) \\
	$W$ = trace width  (in mils) \\
	$H$ = heigh of substrate (in mils) \\
	$\epsilon$ = dielectric constant of material
\end{center}

\begin{figure} [H]
	\centering
	\includegraphics[scale=0.3]{PCB_Trace_Figure.png}
	\caption{PCB Microstrip Impedence Variables\label{fig:pcb-trace}}
\end{figure}

The equation to calculate the impedance is as follows\textsuperscript{\cite{Microstrip}}:
\begin{center}
\large$Z = \frac{Z_0}{2\pi*\sqrt{2}*\sqrt{\epsilon+ 1}} * ln\left(1 + 4*\frac{H}{w_{eff}} * \left(X_1 + X_2 \right )\right )$
\end{center}
where
\begin{center}
	\large $W_{eff} = W + \left(\frac{T}{\pi}\right)*ln\left \{\frac{4*e}{\sqrt{\left(\frac{T}{H}\right)^2 + \left (\frac{T}{W*\pi + 1.1*T*\pi }  \right )^{2}}} \right \}$
	
	\vspace{2.5mm}
	
	$X_1 = \frac{4*\left( 14*\epsilon +8\right)}{11*\epsilon} * \left(\frac{H}{W_{eff}} \right)$
	
	\vspace{2.5mm}
	
	$X_2 = \sqrt{16*\left(\frac{H}{W_{eff}} \right )^2 * \left(\frac{14*\epsilon+8}{11*\epsilon} \right )^2 * \left(\frac{\epsilon + 1}{2*\epsilon} \right ) * \pi^2}$
\end{center}

The ECE parts shop uses 1 oz copper trace and FR4 board material as its substrate \cite{ECE-Electronics-Shop} Assuming these properties have not changed at the time of this design review, then the following values can be used for T, H, Z, $Z_0$ and $\epsilon$ :
\begin{center}
	$Z_0 =$ impedence of free space $\approx$ 120 $\pi \Omega$ \\
	$T = 1.4 $mils\\
	$H =1.6 $mm\\
	$\epsilon = 1.4$ \\
	$Z = 50 \Omega$
\end{center}
The result after plugging in each respective value and solving for W, is that the PCB trace must be \textbf{3.23 mm} wide going from each R.F. module to the antenna.


\normalsize\textbf{R.F. Transmitter/Receiver Range} \\
The range of the R.F. transmitter and receiver is required to reach a distance of 300 meters. The range in kilometers is a function of the following variables:
\begin{center}
$P_T$ = Transmitter Power (dBm) \\
$A_g$ = Total Antenna Gain (dB) \\
$C_l$ = Connection Loss (dB) \\ 
$G_{tot}$ = Total Gain (dB) \\
$R$ = Receiver Sensitivity (dBm) \\
$L$ = Transmission Path Loss (dB) \\
$f_{MHz}$ = Frequency in MHz \\
\end{center}

Transmission Path Loss is the sum of all the antenna R.F. gains and deduction of all possible losses. Assuming a perfect system on the ground without any interference the following equation can be used to calculate the path loss in a transmission:
\begin{center}
	$L = P_T + (A_g - C_l)$
\end{center}
This path loss can be used in conjunction with the following equation to calculate the total range in kilometers\textsuperscript{\cite{RFDistance}}:
\begin{center}
	\large
	$d_{TX->RX} = 10^{\left( \frac{L - 32.45 -20*log(f_{MHz})}{20}\right)}$
\end{center}
The following values were used in this range calculation:
\begin{center}
$P_T$ = -1dBm \\
$A_g$ = 2.15 dBi (quarter-wave monopole) \\
$C_l$ = 0 dB \\ 
$G_{tot}$ =  2.15dB\\
$R$ = -116 dBm \\
$L$ =  120.15\\
\end{center}
Plugging in these values to the equation stated before, the team received a range of \textbf{73.03 km} which well surpasses the requirements of 300 m. \\

\normalsize\textbf{Power Module (Friendly Target Unit)} \\
The following section is intended to backup the design choices made for the power module on the friendly target first shown in Section \ref{section-friendly-interrogator-design}.

The team placed a strict requirement (shown in Section \ref{section-requirements-verification}) regarding the operation time of the friendly target unit (at 8 hours of operation time $\pm$ 5\%).

As with the Interrogator Unit; In order to select parts that satisfied this requirement, the active current consumption on the entire unit must first be calculated. The main power consumption modules on board the friendly target unit are the MSP430F2274, the photoreceiver operational amplifier, and the Linx KH3 R.F. transmitter/decoder. These values were received from each of the respective datasheets. The following table displays the active current consumption of each unit\textsuperscript{\cite{MSP430F2274} \cite{Linx-Receiver} \cite{Laser}}. 


\begin{table}[htbp]
	\centering
	\begin{tabular}{c|c|c}	% ccccccc indicates 7 center aligned columns
		\toprule	% top separator
		Module & Active Current Consumption & Standby Current Consumption\\
		\midrule
		MSP430 & 270 $\mu$A & 0.1 - 0.7 $\mu$A\\ 
		Linx KH3 R.F. Transmitter & 2.7 mA & 1 $\mu$A\\
		Photoreceiver Op Amp & 2.4 mA (max) & 2.4 mA (max) \\
		MAX680 5V to +/- 10V Converter & 500 $\mu$A  & 500 $\mu$A \\
		\bottomrule	% bottom separator
	\end{tabular}%
	\caption{Current Consumption of Target Modules}
	\label{tab:table2}	% this is the label given to the table that can be referenced using \ref{tab:Exp1Part1_7}
\end{table}%

With this information, the maximum possible active current consumption will be:
\begin{center}{I\textsubscript{Total} = $270 \mu A + 2.7 mA + 2.4 mA + 500 \mu A$}\end{center}

As stated for the interrogation unit, the capacity of the Alkaline AA 1.5V battery will be assumed to be 2150 mAh. 

The team is using the AAT1217 step-up converter this boosts the voltage up from 1.5V to 3.3V with a ~75\% efficiency\textsuperscript{\cite{AAT1217}}. The team is using a MAX680 +5V to +/-10V Voltage Converter which has as 85\% power-conversion efficiency \textsuperscript{\cite{MAX680}} . 

The use time can be calculated as follows:

\begin{center} 
	$\textrm{Total Energy} = C_\textrm{battery} * V_\textrm{battery} = 2150 mAh * 1.5V = 3225 mWh$
	
	$\textrm{Total Active Power} = \frac{20V * 2.4 mA}{.85 * .75} + \frac{3.3 V * (0.270 + 2.7 + 0.5) mA}{.75} = 90.5621 mW$
	
	$\textrm{Active Use Time} = \frac{\textrm{Total Energy}}{\textrm{Total Active Power}} \approx \textrm{\textbf{24 hr active use}}$
	
	$\textrm{Total Standby Power} = \frac{20V * 2.4 mA}{.85 * .75} + \frac{3.3 V * (0.270 + .001 + 0.5) mA}{.75} = 78.6865 mW$
	
	$\textrm{Standby Use Time} = \frac{\textrm{Total Energy}}{\textrm{Total Passive Power}} \approx \textrm{\textbf{27 hr standby use}}$
\end{center}

This result shows that a single standard alkaline AA 1.5V disposable battery will be more than sufficient enough to satisfy the requirements of 8 hours of active use time.


\subsubsection{Simulations/Plots}
\hl{PUT SIMULATIONS/PLOTS HERE}


%SECTION - Requirements and Verification
\section{Requirements and Verification} \label{section-requirements-verification}
\hl{PUT REQUIREMENTS/VERIFICATION HERE}

\subsection{Tolerance Analysis} \label{section-tolerance-analysis}
The inherent limitations of laser power for safety means that the performance is in the hands of the photodiode. The starting point for selection criteria was to choose a photodiode type. There are three main photodiode types: normal, PIN, and Avalanche; the team chose PIN photodiodes as they have a high sensitivity and speed. A normal photodiode would not be sensitive enough to register a wide divergence $5mW$ laser, and an Avalanche photodiode requires high voltage. 

The next selection criteria is the material with which the photodiode is made. This includes materials such as Si, InGaAs, and InA. The optimum wavelength is dependent on the material selection.

With photodiodes, Noise-equivalent Power (NEP) is a measure of the incident power required to generate a response signal equal to the noise level of a detector system. Detectivity is the reciprocal of the NEP normalized for the active area of the photodiode.\textsuperscript{\cite{Microphotonics}}. The best photodiode, then, will have the highest detectivity for the visible wavelength.  

\begin{figure} [H]
	\centering
	\includegraphics[scale=0.4]{detectivity-table.png}
	\label{fig:detectivity-table}
	\caption{Specific Detectivity for Photodetector Materials \textsuperscript{\cite{Optical}} \label{fig:detectivity-table}}
\end{figure}

Figure \ref{fig:detectivity-table} illustrates the specific detectivity ranges of photodiodes. The interrogation laser is in the visible range; therefore, the matching photodiode is of type Si. Using this type of photodiode, the detectivity is between $10^{10}$ and $10^{13}$ $\frac{Hz^{\frac{1}{2}}}{W}$. The following calculations will use a conservative value, $10^{12} \frac{Hz^{\frac{1}{2}}}{W}$, as the detectivity. In realty, because the wavelength is less than \SI{1}{\micro\meter}, the detectivity is somewhere between $10^{12}$ and $10^{13}$ $\frac{Hz^{\frac{1}{2}}}{W}$

The equation for NEP from detectivity, $D^*$, and photodiode active area, $A$,  is 
\begin{center}
	{\large $NEP = \frac{\sqrt{A}}{D^*}$}  $[\frac{W}{Hz^{1/2}}]$
\end{center}

The incident irradiance, $E_i$, to cancel noise is
\begin{center}
	{\large $E_i = \frac{NEP * \sqrt{f}}{A} = \frac{\sqrt{Af}}{AD^*} = \frac{f}{D^*\sqrt{A}}$} $[\frac{W}{m^{2}}]$
\end{center}

The NEP measures the incident irradiance to cancel the noise on the photodiode. To register a signal on the MCU, the incident irradiance must be higher than the noise. To be conservative, define the required incident irradiance as 
\begin{center}
	{\large $E_{req} = 2E_i$} $[\frac{W}{m^{2}}]$
\end{center}

Multiplying the area of the laser's spot by the required incident irradiance at the photodiode gives the necessary power. Thus, the radius of the spot in terms of the power of the laser and required incident irradiance at the photodiode is 
\begin{center}
	{\large $r = \sqrt{\frac{P}{\pi E_{req}}}$} $[m]$
\end{center}

Note that the power contained in the laser's spot does not depend on distance from the source, as atmospheric reflection is negligible at $300 m$.

The radius, in terms of the detectivity, frequency, and sensor active area is
\begin{center}
	{\large $r = \sqrt{\frac{PD^*\sqrt{A}}{2 \pi f}}$} $[m]$
\end{center}

The proposal listed $0.8382 m$ as the ideal radius of the laser's spot. Unfortunately, with the $5mW$ red laser, this would require a sensor with a massive active area. The largest sensor the team could find, at a reasonable price, has a $8.53 mm^2$ active area. 

For the $8.53 mm^2$ active area photodiode operating at $650 nm = 4.61219 × 10^{14} Hz$ 
\begin{center}
	$r = 0.608721 m \approx 33 cm$
\end{center}

Refining the proposal requirements, the team has set a new requirement of a minimum $20 cm$ laser spot radius, making the diameter of the beam $0.4m$ at distances of $50m, 150m,$ and $300m$ with optical adjustment. 

Capturing these requirements, the team must transmit a signal to a photoreceiver at the following ranges:
\begin{itemize}
	\item Short Range (0 - 50 m)
	\item Medium Range (50 - 150 m)
	\item Long Range (150 - 300 m)
\end{itemize}

The team will verify this requirement with the following procedure:

\begin{enumerate}
	\item Measure and label each distance on ground surface.
	\item Setup target at desired location (1 out of 3 locations), being a 1.5 x 1.5 meter white, flat, square surface and measure out a circle with 20 cm radius, centered at the origin of the surface.
	\item Place 4 photoreceivers on outer perimeter of 20cm circle.
	\item Connect multimeter to output of photoreceivers
	\item Setup laser on stable surface at source, and connect power line to 130 ohm resistor and ground line to GND.
	\item Place string at base of laser and guide string to the origin of the target surface. This is to ensure that the laser is in line with the target and accurately hits the target.
	\item Note the value on the multimeter
	\item Power on laser by providing it a constant 3.3V power source.
	\item Ensure the multimeter does not fall below the value noted in (7) for a period of 5 seconds. 
\end{enumerate}

\subsection{Safety} \label{section-safety-ethics}
\normalsize\textbf{Laser Safety} \\
To achieve the laser beam diameter at $300 m$ associated in the proposal, a Class 3B laser would be required. In the State of Illinois, a Class 3B laser must be registered with the Division of Nuclear Safety in the Illinois Emergency Management Agency. The 3B laser would also present a significant viewing hazard; especially in an application where the laser is intended to be pointed at people. 

For the reasons stated above, the team will instead use a $5mW$ visible red laser. $5mW$ visible lasers have a low chance of injuring the eye, as the blinking reflex will save a victim from permanent damage; as opposed to IR lasers which can go unnoticed for several seconds. 

The following is a calculation for the nominal ocular hazard distance (NOHD) of the laser, as defined by the ANSI Standard\textsuperscript{\cite{ANSI}}.

The maximum permissible exposure (MPE), as defined by the ANSI Standard \textsuperscript{\cite{ANSI}} is the highest power or energy density of a light source that is considered safe, i.e. that has a negligible probability for creating damage. This MPE for a pulsing laser is calculated as the minimum of the following three rules:

\begin{enumerate}
	\item Any single pulse in the train must not exceed the MPE for the pulse exposure time.
	\item The exposure from any group of pulses delivered in time T must not exceed the MPE for
	time T, where T is 0.25 seconds (from the blinking reflex), for a visible laser. 
	\item For thermal injury, the exposure for any single pulse within a group of pulses must not
	exceed the single-pulse MPE multiplied by a multiple-pulse correction factor
\end{enumerate}

The laser will pulse at a rate of $40 kHz$. Assuming at most a 50\% duty cycle, each pulse will be of max length $1.25*10^{-5} s$. The divergence of the beam is smallest for the longest range; a lower divergence is more restrictive in terms of safety, so this calculation uses $300m$. 

At $5mW$ with a pulse width of $1.25*10^{-5}$, the power of the laser is $6.25*10^{-8} J$. 

ANSI defines several constants for use in the calculation of laser safety. The relevant constant for these calculations is the constant $C_6$. This is defined as 
\begin{center}
	\large
	$C_6 =$
	$\frac{\theta}{1.5}$ for $1.5 \leq \theta \leq 100$\\
	$C_6 = 1$ for $\theta < 1.5, \theta > 100$
\end{center}

Using trigonometry, the divergence angle, $\theta$, for the laser is 
\begin{center}
	\large
	$Tan^{-1}(\frac{r}{300})* 1000$ $[mrad]$
\end{center}

Following the ANSI Standard \cite{ANSI}, the Rule 1 calculation is 
\begin{center}
	\large
    $R_1 = 5*10^{-3} * C_6$
\end{center}

The Rule 2 calculation is
\begin{center}
	\large
	$R_2 = 18 (T)^{0.75}$
\end{center}

The Rule 3 calculation is
\begin{center}
	\large
	$R_3 = R1(T*f)^{0.25}$
\end{center}

The most restrictive rule defines the MPE 
\begin{center}
	\large
	$MPE = min(R_1, R_2, R_3)$
\end{center}

The MPE, then, is\\
{\large $min($}
\begin{center}
	\large
	$ 5*10^{-3} * Tan^{-1}(\frac{.5}{300})* 1000$\\
	$18 (.25)^{0.75}$ \\
	$0.00833333 (0.25*40000)^{0.25}$
\end{center}
{\large $)$}

This gives 
\begin{center}
	\large
	$MPE = min(0.00833333, 6.36396, 0.0833333) = 0.00833333 [\frac{J}{m^2}]$
\end{center}

The NOHD is defined as (with $\theta$ in terms of $rad$, not $mrad$)
\begin{center}
	\large
$ \frac{\sqrt{\frac{4 * P}{\pi * MPE}} - 2w}{\theta}$
\end{center}

Where P is the power of the beam ($6.25*10^{-8} J$) and $w$ is the waist of the beam, $0.5mm$. This gives an NOHD of 
\begin{center}
	\large
	$ \frac{\sqrt{\frac{4 * 6.25*10^{-8} }{\pi * 0.00833333}} - 2*0.0005}{Tan^{-1}(\frac{.5}{300})} = 1.25 m$
\end{center}

The team will take precautions to avoid eye contact with the laser within $1.25m$ of the source. If it is absolutely necessary to work with the laser powered on and a person within $1.25m$ of the laser, the person will be required to wear protective eye wear. 

The risk of eye damage is mitigated by the fact that the laser is both visible, and not always powered on. 

\normalsize\textbf{Electrical Safety} \\
The majority of the components operate at less than $5V$, which does not present a significant risk. Two components, however, operate with a voltage differential of $20V$. This is the $5V$ to $10V$ converter and the Operational Amplifier used for the photoreceiver. The team will exercise caution when working with these units; using a multimeter to verify the part is powered off before making contact. 

The team will also follow normal battery safety procedures with all batteries in use; ensuring that they are stored at a safe temperature, are not leaking battery acid or damaged, and are not shorted. 


\subsection{Ethical Issues}
This project has several ethical issues that can be addressed by the IEEE Code of Conduct. Specifically, numbers 1, 2, 3, 5, 6, 7, and 9 are the most important items that pertain to the Infantry I.F.F. System the team is building this semester. 
\begin{enumerate}
\item to accept responsibility in making decisions consistent with the safety, health, and welfare of the public, and to disclose promptly factors that might endanger the public or the environment;
\item to avoid real or perceived conflicts of interest whenever possible, and to disclose them to affected parties when they do exist;
\item to be honest and realistic in stating claims or estimates based on available data;  
\skipitems{1}
\item to improve the understanding of technology; its appropriate application, and potential consequences;  
\item to maintain and improve our technical competence and to undertake technological tasks for others only if qualified by training or experience, or after full disclosure of pertinent limitations;  
\item to seek, accept, and offer honest criticism of technical work, to acknowledge and correct errors, and to credit properly the contributions of others;  
\skipitems{1}
\item to avoid injuring others, their property, reputation, or employment by false or malicious action;  
\end{enumerate}

%SECTION - Cost and Schedule
\section{Cost and Schedule}

\subsection{Cost Analysis}
The labor cost was calculated as follows:

\begin{center}
	Labor Cost = Worker Salary (\$/hour) x 2.5 x Time (Hours) Invested In Project
\end{center}

Most parts were gathered using the resources found on Digi-Key Electronics and a table of these parts can be found in the Appendix in Figure \ref{fig:parts-cost-list}. \\

\subsection{Schedule}

\begin{figure} [H]
	\centering
	\includegraphics[scale=0.63]{Schedule_Extended.png}
	\caption{Schedule\label{fig:friendly-interrogator-software-flowchart}}
\end{figure}


%SECTION - References
\clearpage
\bibliographystyle{IEEE_ECE}
% include the BibTex file here to build reference
\bibliography{Citations}\addcontentsline{toc}{section}{Reference}
\clearpage
\section*{Appendix}
\begin{figure} [H]
	\centering
	\includepdf[page=1, scale=0.85] {Design-Review-Requirements-Verification.pdf}
\end{figure}
\clearpage
\begin{figure} [H]
	\centering
	\includepdf[page=2, scale=0.85] {Design-Review-Requirements-Verification.pdf}
\end{figure}
\clearpage
\begin{figure} [H]
	\centering
	\includegraphics[scale=0.450]{Friendly_Interrogator_Software_Flowchart.pdf}
	\caption{Flowchart for Functionality\label{fig:friendly-interrogator-software-flowchart}}
\end{figure}

<<<<<<< HEAD
\begin{figure} [H]
	\centering
	\includegraphics[scale=0.400]{Friendly_Target_Software_Flowchart.pdf}
	\caption{Flowchart for Functionality\label{fig:friendly-target-software-flowchart}}
\end{figure}

\begin{figure} [H]
	\centering
	\includegraphics[scale = 0.9, angle=90]{Parts_and_Labor_Cost.pdf}
	\caption{Parts and Labor Estimate\label{fig:parts-cost-list}}
\end{figure}

\end{spacing}
\end{document}

